
% Default to the notebook output style

    


% Inherit from the specified cell style.




    
\documentclass[11pt]{article}

    
    
    \usepackage[T1]{fontenc}
    % Nicer default font (+ math font) than Computer Modern for most use cases
    \usepackage{mathpazo}

    % Basic figure setup, for now with no caption control since it's done
    % automatically by Pandoc (which extracts ![](path) syntax from Markdown).
    \usepackage{graphicx}
    % We will generate all images so they have a width \maxwidth. This means
    % that they will get their normal width if they fit onto the page, but
    % are scaled down if they would overflow the margins.
    \makeatletter
    \def\maxwidth{\ifdim\Gin@nat@width>\linewidth\linewidth
    \else\Gin@nat@width\fi}
    \makeatother
    \let\Oldincludegraphics\includegraphics
    % Set max figure width to be 80% of text width, for now hardcoded.
    \renewcommand{\includegraphics}[1]{\Oldincludegraphics[width=.8\maxwidth]{#1}}
    % Ensure that by default, figures have no caption (until we provide a
    % proper Figure object with a Caption API and a way to capture that
    % in the conversion process - todo).
    \usepackage{caption}
    \DeclareCaptionLabelFormat{nolabel}{}
    \captionsetup{labelformat=nolabel}

    \usepackage{adjustbox} % Used to constrain images to a maximum size 
    \usepackage{xcolor} % Allow colors to be defined
    \usepackage{enumerate} % Needed for markdown enumerations to work
    \usepackage{geometry} % Used to adjust the document margins
    \usepackage{amsmath} % Equations
    \usepackage{amssymb} % Equations
    \usepackage{textcomp} % defines textquotesingle
    % Hack from http://tex.stackexchange.com/a/47451/13684:
    \AtBeginDocument{%
        \def\PYZsq{\textquotesingle}% Upright quotes in Pygmentized code
    }
    \usepackage{upquote} % Upright quotes for verbatim code
    \usepackage{eurosym} % defines \euro
    \usepackage[mathletters]{ucs} % Extended unicode (utf-8) support
    \usepackage[utf8x]{inputenc} % Allow utf-8 characters in the tex document
    \usepackage{fancyvrb} % verbatim replacement that allows latex
    \usepackage{grffile} % extends the file name processing of package graphics 
                         % to support a larger range 
    % The hyperref package gives us a pdf with properly built
    % internal navigation ('pdf bookmarks' for the table of contents,
    % internal cross-reference links, web links for URLs, etc.)
    \usepackage{hyperref}
    \usepackage{longtable} % longtable support required by pandoc >1.10
    \usepackage{booktabs}  % table support for pandoc > 1.12.2
    \usepackage[inline]{enumitem} % IRkernel/repr support (it uses the enumerate* environment)
    \usepackage[normalem]{ulem} % ulem is needed to support strikethroughs (\sout)
                                % normalem makes italics be italics, not underlines
    

    
    
    % Colors for the hyperref package
    \definecolor{urlcolor}{rgb}{0,.145,.698}
    \definecolor{linkcolor}{rgb}{.71,0.21,0.01}
    \definecolor{citecolor}{rgb}{.12,.54,.11}

    % ANSI colors
    \definecolor{ansi-black}{HTML}{3E424D}
    \definecolor{ansi-black-intense}{HTML}{282C36}
    \definecolor{ansi-red}{HTML}{E75C58}
    \definecolor{ansi-red-intense}{HTML}{B22B31}
    \definecolor{ansi-green}{HTML}{00A250}
    \definecolor{ansi-green-intense}{HTML}{007427}
    \definecolor{ansi-yellow}{HTML}{DDB62B}
    \definecolor{ansi-yellow-intense}{HTML}{B27D12}
    \definecolor{ansi-blue}{HTML}{208FFB}
    \definecolor{ansi-blue-intense}{HTML}{0065CA}
    \definecolor{ansi-magenta}{HTML}{D160C4}
    \definecolor{ansi-magenta-intense}{HTML}{A03196}
    \definecolor{ansi-cyan}{HTML}{60C6C8}
    \definecolor{ansi-cyan-intense}{HTML}{258F8F}
    \definecolor{ansi-white}{HTML}{C5C1B4}
    \definecolor{ansi-white-intense}{HTML}{A1A6B2}

    % commands and environments needed by pandoc snippets
    % extracted from the output of `pandoc -s`
    \providecommand{\tightlist}{%
      \setlength{\itemsep}{0pt}\setlength{\parskip}{0pt}}
    \DefineVerbatimEnvironment{Highlighting}{Verbatim}{commandchars=\\\{\}}
    % Add ',fontsize=\small' for more characters per line
    \newenvironment{Shaded}{}{}
    \newcommand{\KeywordTok}[1]{\textcolor[rgb]{0.00,0.44,0.13}{\textbf{{#1}}}}
    \newcommand{\DataTypeTok}[1]{\textcolor[rgb]{0.56,0.13,0.00}{{#1}}}
    \newcommand{\DecValTok}[1]{\textcolor[rgb]{0.25,0.63,0.44}{{#1}}}
    \newcommand{\BaseNTok}[1]{\textcolor[rgb]{0.25,0.63,0.44}{{#1}}}
    \newcommand{\FloatTok}[1]{\textcolor[rgb]{0.25,0.63,0.44}{{#1}}}
    \newcommand{\CharTok}[1]{\textcolor[rgb]{0.25,0.44,0.63}{{#1}}}
    \newcommand{\StringTok}[1]{\textcolor[rgb]{0.25,0.44,0.63}{{#1}}}
    \newcommand{\CommentTok}[1]{\textcolor[rgb]{0.38,0.63,0.69}{\textit{{#1}}}}
    \newcommand{\OtherTok}[1]{\textcolor[rgb]{0.00,0.44,0.13}{{#1}}}
    \newcommand{\AlertTok}[1]{\textcolor[rgb]{1.00,0.00,0.00}{\textbf{{#1}}}}
    \newcommand{\FunctionTok}[1]{\textcolor[rgb]{0.02,0.16,0.49}{{#1}}}
    \newcommand{\RegionMarkerTok}[1]{{#1}}
    \newcommand{\ErrorTok}[1]{\textcolor[rgb]{1.00,0.00,0.00}{\textbf{{#1}}}}
    \newcommand{\NormalTok}[1]{{#1}}
    
    % Additional commands for more recent versions of Pandoc
    \newcommand{\ConstantTok}[1]{\textcolor[rgb]{0.53,0.00,0.00}{{#1}}}
    \newcommand{\SpecialCharTok}[1]{\textcolor[rgb]{0.25,0.44,0.63}{{#1}}}
    \newcommand{\VerbatimStringTok}[1]{\textcolor[rgb]{0.25,0.44,0.63}{{#1}}}
    \newcommand{\SpecialStringTok}[1]{\textcolor[rgb]{0.73,0.40,0.53}{{#1}}}
    \newcommand{\ImportTok}[1]{{#1}}
    \newcommand{\DocumentationTok}[1]{\textcolor[rgb]{0.73,0.13,0.13}{\textit{{#1}}}}
    \newcommand{\AnnotationTok}[1]{\textcolor[rgb]{0.38,0.63,0.69}{\textbf{\textit{{#1}}}}}
    \newcommand{\CommentVarTok}[1]{\textcolor[rgb]{0.38,0.63,0.69}{\textbf{\textit{{#1}}}}}
    \newcommand{\VariableTok}[1]{\textcolor[rgb]{0.10,0.09,0.49}{{#1}}}
    \newcommand{\ControlFlowTok}[1]{\textcolor[rgb]{0.00,0.44,0.13}{\textbf{{#1}}}}
    \newcommand{\OperatorTok}[1]{\textcolor[rgb]{0.40,0.40,0.40}{{#1}}}
    \newcommand{\BuiltInTok}[1]{{#1}}
    \newcommand{\ExtensionTok}[1]{{#1}}
    \newcommand{\PreprocessorTok}[1]{\textcolor[rgb]{0.74,0.48,0.00}{{#1}}}
    \newcommand{\AttributeTok}[1]{\textcolor[rgb]{0.49,0.56,0.16}{{#1}}}
    \newcommand{\InformationTok}[1]{\textcolor[rgb]{0.38,0.63,0.69}{\textbf{\textit{{#1}}}}}
    \newcommand{\WarningTok}[1]{\textcolor[rgb]{0.38,0.63,0.69}{\textbf{\textit{{#1}}}}}
    
    
    % Define a nice break command that doesn't care if a line doesn't already
    % exist.
    \def\br{\hspace*{\fill} \\* }
    % Math Jax compatability definitions
    \def\gt{>}
    \def\lt{<}
    % Document parameters
    \title{proj2\_part1}
    
    
    

    % Pygments definitions
    
\makeatletter
\def\PY@reset{\let\PY@it=\relax \let\PY@bf=\relax%
    \let\PY@ul=\relax \let\PY@tc=\relax%
    \let\PY@bc=\relax \let\PY@ff=\relax}
\def\PY@tok#1{\csname PY@tok@#1\endcsname}
\def\PY@toks#1+{\ifx\relax#1\empty\else%
    \PY@tok{#1}\expandafter\PY@toks\fi}
\def\PY@do#1{\PY@bc{\PY@tc{\PY@ul{%
    \PY@it{\PY@bf{\PY@ff{#1}}}}}}}
\def\PY#1#2{\PY@reset\PY@toks#1+\relax+\PY@do{#2}}

\expandafter\def\csname PY@tok@w\endcsname{\def\PY@tc##1{\textcolor[rgb]{0.73,0.73,0.73}{##1}}}
\expandafter\def\csname PY@tok@c\endcsname{\let\PY@it=\textit\def\PY@tc##1{\textcolor[rgb]{0.25,0.50,0.50}{##1}}}
\expandafter\def\csname PY@tok@cp\endcsname{\def\PY@tc##1{\textcolor[rgb]{0.74,0.48,0.00}{##1}}}
\expandafter\def\csname PY@tok@k\endcsname{\let\PY@bf=\textbf\def\PY@tc##1{\textcolor[rgb]{0.00,0.50,0.00}{##1}}}
\expandafter\def\csname PY@tok@kp\endcsname{\def\PY@tc##1{\textcolor[rgb]{0.00,0.50,0.00}{##1}}}
\expandafter\def\csname PY@tok@kt\endcsname{\def\PY@tc##1{\textcolor[rgb]{0.69,0.00,0.25}{##1}}}
\expandafter\def\csname PY@tok@o\endcsname{\def\PY@tc##1{\textcolor[rgb]{0.40,0.40,0.40}{##1}}}
\expandafter\def\csname PY@tok@ow\endcsname{\let\PY@bf=\textbf\def\PY@tc##1{\textcolor[rgb]{0.67,0.13,1.00}{##1}}}
\expandafter\def\csname PY@tok@nb\endcsname{\def\PY@tc##1{\textcolor[rgb]{0.00,0.50,0.00}{##1}}}
\expandafter\def\csname PY@tok@nf\endcsname{\def\PY@tc##1{\textcolor[rgb]{0.00,0.00,1.00}{##1}}}
\expandafter\def\csname PY@tok@nc\endcsname{\let\PY@bf=\textbf\def\PY@tc##1{\textcolor[rgb]{0.00,0.00,1.00}{##1}}}
\expandafter\def\csname PY@tok@nn\endcsname{\let\PY@bf=\textbf\def\PY@tc##1{\textcolor[rgb]{0.00,0.00,1.00}{##1}}}
\expandafter\def\csname PY@tok@ne\endcsname{\let\PY@bf=\textbf\def\PY@tc##1{\textcolor[rgb]{0.82,0.25,0.23}{##1}}}
\expandafter\def\csname PY@tok@nv\endcsname{\def\PY@tc##1{\textcolor[rgb]{0.10,0.09,0.49}{##1}}}
\expandafter\def\csname PY@tok@no\endcsname{\def\PY@tc##1{\textcolor[rgb]{0.53,0.00,0.00}{##1}}}
\expandafter\def\csname PY@tok@nl\endcsname{\def\PY@tc##1{\textcolor[rgb]{0.63,0.63,0.00}{##1}}}
\expandafter\def\csname PY@tok@ni\endcsname{\let\PY@bf=\textbf\def\PY@tc##1{\textcolor[rgb]{0.60,0.60,0.60}{##1}}}
\expandafter\def\csname PY@tok@na\endcsname{\def\PY@tc##1{\textcolor[rgb]{0.49,0.56,0.16}{##1}}}
\expandafter\def\csname PY@tok@nt\endcsname{\let\PY@bf=\textbf\def\PY@tc##1{\textcolor[rgb]{0.00,0.50,0.00}{##1}}}
\expandafter\def\csname PY@tok@nd\endcsname{\def\PY@tc##1{\textcolor[rgb]{0.67,0.13,1.00}{##1}}}
\expandafter\def\csname PY@tok@s\endcsname{\def\PY@tc##1{\textcolor[rgb]{0.73,0.13,0.13}{##1}}}
\expandafter\def\csname PY@tok@sd\endcsname{\let\PY@it=\textit\def\PY@tc##1{\textcolor[rgb]{0.73,0.13,0.13}{##1}}}
\expandafter\def\csname PY@tok@si\endcsname{\let\PY@bf=\textbf\def\PY@tc##1{\textcolor[rgb]{0.73,0.40,0.53}{##1}}}
\expandafter\def\csname PY@tok@se\endcsname{\let\PY@bf=\textbf\def\PY@tc##1{\textcolor[rgb]{0.73,0.40,0.13}{##1}}}
\expandafter\def\csname PY@tok@sr\endcsname{\def\PY@tc##1{\textcolor[rgb]{0.73,0.40,0.53}{##1}}}
\expandafter\def\csname PY@tok@ss\endcsname{\def\PY@tc##1{\textcolor[rgb]{0.10,0.09,0.49}{##1}}}
\expandafter\def\csname PY@tok@sx\endcsname{\def\PY@tc##1{\textcolor[rgb]{0.00,0.50,0.00}{##1}}}
\expandafter\def\csname PY@tok@m\endcsname{\def\PY@tc##1{\textcolor[rgb]{0.40,0.40,0.40}{##1}}}
\expandafter\def\csname PY@tok@gh\endcsname{\let\PY@bf=\textbf\def\PY@tc##1{\textcolor[rgb]{0.00,0.00,0.50}{##1}}}
\expandafter\def\csname PY@tok@gu\endcsname{\let\PY@bf=\textbf\def\PY@tc##1{\textcolor[rgb]{0.50,0.00,0.50}{##1}}}
\expandafter\def\csname PY@tok@gd\endcsname{\def\PY@tc##1{\textcolor[rgb]{0.63,0.00,0.00}{##1}}}
\expandafter\def\csname PY@tok@gi\endcsname{\def\PY@tc##1{\textcolor[rgb]{0.00,0.63,0.00}{##1}}}
\expandafter\def\csname PY@tok@gr\endcsname{\def\PY@tc##1{\textcolor[rgb]{1.00,0.00,0.00}{##1}}}
\expandafter\def\csname PY@tok@ge\endcsname{\let\PY@it=\textit}
\expandafter\def\csname PY@tok@gs\endcsname{\let\PY@bf=\textbf}
\expandafter\def\csname PY@tok@gp\endcsname{\let\PY@bf=\textbf\def\PY@tc##1{\textcolor[rgb]{0.00,0.00,0.50}{##1}}}
\expandafter\def\csname PY@tok@go\endcsname{\def\PY@tc##1{\textcolor[rgb]{0.53,0.53,0.53}{##1}}}
\expandafter\def\csname PY@tok@gt\endcsname{\def\PY@tc##1{\textcolor[rgb]{0.00,0.27,0.87}{##1}}}
\expandafter\def\csname PY@tok@err\endcsname{\def\PY@bc##1{\setlength{\fboxsep}{0pt}\fcolorbox[rgb]{1.00,0.00,0.00}{1,1,1}{\strut ##1}}}
\expandafter\def\csname PY@tok@kc\endcsname{\let\PY@bf=\textbf\def\PY@tc##1{\textcolor[rgb]{0.00,0.50,0.00}{##1}}}
\expandafter\def\csname PY@tok@kd\endcsname{\let\PY@bf=\textbf\def\PY@tc##1{\textcolor[rgb]{0.00,0.50,0.00}{##1}}}
\expandafter\def\csname PY@tok@kn\endcsname{\let\PY@bf=\textbf\def\PY@tc##1{\textcolor[rgb]{0.00,0.50,0.00}{##1}}}
\expandafter\def\csname PY@tok@kr\endcsname{\let\PY@bf=\textbf\def\PY@tc##1{\textcolor[rgb]{0.00,0.50,0.00}{##1}}}
\expandafter\def\csname PY@tok@bp\endcsname{\def\PY@tc##1{\textcolor[rgb]{0.00,0.50,0.00}{##1}}}
\expandafter\def\csname PY@tok@fm\endcsname{\def\PY@tc##1{\textcolor[rgb]{0.00,0.00,1.00}{##1}}}
\expandafter\def\csname PY@tok@vc\endcsname{\def\PY@tc##1{\textcolor[rgb]{0.10,0.09,0.49}{##1}}}
\expandafter\def\csname PY@tok@vg\endcsname{\def\PY@tc##1{\textcolor[rgb]{0.10,0.09,0.49}{##1}}}
\expandafter\def\csname PY@tok@vi\endcsname{\def\PY@tc##1{\textcolor[rgb]{0.10,0.09,0.49}{##1}}}
\expandafter\def\csname PY@tok@vm\endcsname{\def\PY@tc##1{\textcolor[rgb]{0.10,0.09,0.49}{##1}}}
\expandafter\def\csname PY@tok@sa\endcsname{\def\PY@tc##1{\textcolor[rgb]{0.73,0.13,0.13}{##1}}}
\expandafter\def\csname PY@tok@sb\endcsname{\def\PY@tc##1{\textcolor[rgb]{0.73,0.13,0.13}{##1}}}
\expandafter\def\csname PY@tok@sc\endcsname{\def\PY@tc##1{\textcolor[rgb]{0.73,0.13,0.13}{##1}}}
\expandafter\def\csname PY@tok@dl\endcsname{\def\PY@tc##1{\textcolor[rgb]{0.73,0.13,0.13}{##1}}}
\expandafter\def\csname PY@tok@s2\endcsname{\def\PY@tc##1{\textcolor[rgb]{0.73,0.13,0.13}{##1}}}
\expandafter\def\csname PY@tok@sh\endcsname{\def\PY@tc##1{\textcolor[rgb]{0.73,0.13,0.13}{##1}}}
\expandafter\def\csname PY@tok@s1\endcsname{\def\PY@tc##1{\textcolor[rgb]{0.73,0.13,0.13}{##1}}}
\expandafter\def\csname PY@tok@mb\endcsname{\def\PY@tc##1{\textcolor[rgb]{0.40,0.40,0.40}{##1}}}
\expandafter\def\csname PY@tok@mf\endcsname{\def\PY@tc##1{\textcolor[rgb]{0.40,0.40,0.40}{##1}}}
\expandafter\def\csname PY@tok@mh\endcsname{\def\PY@tc##1{\textcolor[rgb]{0.40,0.40,0.40}{##1}}}
\expandafter\def\csname PY@tok@mi\endcsname{\def\PY@tc##1{\textcolor[rgb]{0.40,0.40,0.40}{##1}}}
\expandafter\def\csname PY@tok@il\endcsname{\def\PY@tc##1{\textcolor[rgb]{0.40,0.40,0.40}{##1}}}
\expandafter\def\csname PY@tok@mo\endcsname{\def\PY@tc##1{\textcolor[rgb]{0.40,0.40,0.40}{##1}}}
\expandafter\def\csname PY@tok@ch\endcsname{\let\PY@it=\textit\def\PY@tc##1{\textcolor[rgb]{0.25,0.50,0.50}{##1}}}
\expandafter\def\csname PY@tok@cm\endcsname{\let\PY@it=\textit\def\PY@tc##1{\textcolor[rgb]{0.25,0.50,0.50}{##1}}}
\expandafter\def\csname PY@tok@cpf\endcsname{\let\PY@it=\textit\def\PY@tc##1{\textcolor[rgb]{0.25,0.50,0.50}{##1}}}
\expandafter\def\csname PY@tok@c1\endcsname{\let\PY@it=\textit\def\PY@tc##1{\textcolor[rgb]{0.25,0.50,0.50}{##1}}}
\expandafter\def\csname PY@tok@cs\endcsname{\let\PY@it=\textit\def\PY@tc##1{\textcolor[rgb]{0.25,0.50,0.50}{##1}}}

\def\PYZbs{\char`\\}
\def\PYZus{\char`\_}
\def\PYZob{\char`\{}
\def\PYZcb{\char`\}}
\def\PYZca{\char`\^}
\def\PYZam{\char`\&}
\def\PYZlt{\char`\<}
\def\PYZgt{\char`\>}
\def\PYZsh{\char`\#}
\def\PYZpc{\char`\%}
\def\PYZdl{\char`\$}
\def\PYZhy{\char`\-}
\def\PYZsq{\char`\'}
\def\PYZdq{\char`\"}
\def\PYZti{\char`\~}
% for compatibility with earlier versions
\def\PYZat{@}
\def\PYZlb{[}
\def\PYZrb{]}
\makeatother


    % Exact colors from NB
    \definecolor{incolor}{rgb}{0.0, 0.0, 0.5}
    \definecolor{outcolor}{rgb}{0.545, 0.0, 0.0}



    
    % Prevent overflowing lines due to hard-to-break entities
    \sloppy 
    % Setup hyperref package
    \hypersetup{
      breaklinks=true,  % so long urls are correctly broken across lines
      colorlinks=true,
      urlcolor=urlcolor,
      linkcolor=linkcolor,
      citecolor=citecolor,
      }
    % Slightly bigger margins than the latex defaults
    
    \geometry{verbose,tmargin=1in,bmargin=1in,lmargin=1in,rmargin=1in}
    
    

    \begin{document}
    
    
    \maketitle
    
    

    
    Before you turn in the homework, make sure everything runs as expected.
To do so, select \textbf{Kernel}\(\rightarrow\)\textbf{Restart \& Run
All} in the toolbar above. Remember to submit both on \textbf{DataHub}
and \textbf{Gradescope}.

Please fill in your name and include a list of your collaborators below.

    \begin{Verbatim}[commandchars=\\\{\}]
{\color{incolor}In [{\color{incolor}1}]:} \PY{n}{NAME} \PY{o}{=} \PY{l+s+s2}{\PYZdq{}}\PY{l+s+s2}{Timlan Wong}\PY{l+s+s2}{\PYZdq{}}
        \PY{n}{COLLABORATORS} \PY{o}{=} \PY{l+s+s2}{\PYZdq{}}\PY{l+s+s2}{\PYZdq{}}
\end{Verbatim}


    \begin{center}\rule{0.5\linewidth}{\linethickness}\end{center}

    \section{Project 2: NYC Taxi Rides}\label{project-2-nyc-taxi-rides}

\section{Part 1: Data Wrangling}\label{part-1-data-wrangling}

In this notebook, we will first query a database to fetch our data and
generate training and test sets.

    \section{Imports}\label{imports}

    \begin{Verbatim}[commandchars=\\\{\}]
{\color{incolor}In [{\color{incolor}2}]:} \PY{k+kn}{import} \PY{n+nn}{os}
        \PY{k+kn}{import} \PY{n+nn}{pandas} \PY{k}{as} \PY{n+nn}{pd}
        \PY{k+kn}{import} \PY{n+nn}{numpy} \PY{k}{as} \PY{n+nn}{np}
        \PY{k+kn}{from} \PY{n+nn}{pathlib} \PY{k}{import} \PY{n}{Path}
        \PY{k+kn}{from} \PY{n+nn}{sqlalchemy} \PY{k}{import} \PY{n}{create\PYZus{}engine}
        \PY{k+kn}{from} \PY{n+nn}{utils} \PY{k}{import} \PY{n}{timeit}
\end{Verbatim}


    \subsection{SQLite}\label{sqlite}

\href{https://www.sqlite.org/whentouse.html}{SQLite} is a SQL database
engine that excels at managing data stored locally in a file. We will be
using SQLite to query for our data. First let's check that our database
is accessible and set up properly. Run the following line to make sure
the data is there and pay attention to how big the data is.

In practice, data is stored in a distributed SQL database that spans
machines (e.g.
\href{https://stackoverflow.com/questions/20030436/what-is-hive-is-it-a-database}{Hive})
or even continents (e.g.
\href{https://en.wikipedia.org/wiki/Spanner_(database)}{Spanner}).
However, how you query the data will remain the same: the SQL language.

    \begin{Verbatim}[commandchars=\\\{\}]
{\color{incolor}In [{\color{incolor}3}]:} \PY{o}{!}ls \PYZhy{}lh /srv/db/taxi\PYZus{}2016\PYZus{}student\PYZus{}small.sqlite
\end{Verbatim}


    \begin{Verbatim}[commandchars=\\\{\}]
-rw-r--r-- 1 root root 2.1G Nov  7 04:43 /srv/db/taxi\_2016\_student\_small.sqlite

    \end{Verbatim}

    Running this line will connect to SQLite engine and test the connection
by printing out the total number of rows.

    \begin{Verbatim}[commandchars=\\\{\}]
{\color{incolor}In [{\color{incolor}4}]:} \PY{n}{DB\PYZus{}URI} \PY{o}{=} \PY{l+s+s2}{\PYZdq{}}\PY{l+s+s2}{sqlite:////srv/db/taxi\PYZus{}2016\PYZus{}student\PYZus{}small.sqlite}\PY{l+s+s2}{\PYZdq{}}
        \PY{n}{TABLE\PYZus{}NAME} \PY{o}{=} \PY{l+s+s2}{\PYZdq{}}\PY{l+s+s2}{taxi}\PY{l+s+s2}{\PYZdq{}}
        
        \PY{n}{sql\PYZus{}engine} \PY{o}{=} \PY{n}{create\PYZus{}engine}\PY{p}{(}\PY{n}{DB\PYZus{}URI}\PY{p}{)}
        \PY{k}{with} \PY{n}{timeit}\PY{p}{(}\PY{p}{)}\PY{p}{:}
            \PY{n+nb}{print}\PY{p}{(}\PY{n}{f}\PY{l+s+s2}{\PYZdq{}}\PY{l+s+s2}{Table }\PY{l+s+si}{\PYZob{}TABLE\PYZus{}NAME\PYZcb{}}\PY{l+s+s2}{ has }\PY{l+s+s2}{\PYZob{}}\PY{l+s+s2}{sql\PYZus{}engine.execute(f}\PY{l+s+s2}{\PYZsq{}}\PY{l+s+s2}{SELECT COUNT(*) FROM }\PY{l+s+si}{\PYZob{}TABLE\PYZus{}NAME\PYZcb{}}\PY{l+s+s2}{\PYZsq{}}\PY{l+s+s2}{).first()[0]\PYZcb{} rows!}\PY{l+s+s2}{\PYZdq{}}\PY{p}{)}
\end{Verbatim}


    \begin{Verbatim}[commandchars=\\\{\}]
Table taxi has 15000000 rows!
0.93 s elapsed

    \end{Verbatim}

    Quick note: One piece of syntax above that you may not be familiar with
is the Python \href{https://realpython.com/python-f-strings/}{f-string},
a relatively new feature to the language.

Basically, it automatically replaces text inside curly braces with the
results of the given expression. For example:

    \begin{Verbatim}[commandchars=\\\{\}]
{\color{incolor}In [{\color{incolor}5}]:} \PY{n}{bloop} \PY{o}{=} \PY{l+s+s2}{\PYZdq{}}\PY{l+s+s2}{wet egg}\PY{l+s+s2}{\PYZdq{}}
        \PY{n+nb}{print}\PY{p}{(}\PY{n}{f}\PY{l+s+s2}{\PYZdq{}}\PY{l+s+si}{\PYZob{}bloop\PYZcb{}}\PY{l+s+s2}{ gets replaced, oh also }\PY{l+s+s2}{\PYZob{}}\PY{l+s+s2}{3 + 5\PYZcb{}.}\PY{l+s+s2}{\PYZdq{}}\PY{p}{)}
\end{Verbatim}


    \begin{Verbatim}[commandchars=\\\{\}]
wet egg gets replaced, oh also 8.

    \end{Verbatim}

    \section{NYC Taxi Data}\label{nyc-taxi-data}

We are working with a much larger dataset (15,000,000 rows!), larger
than anything we have worked with before. If you are not careful in
writing your queries, you may crash your kernel. Please do not
\texttt{"SELECT\ *\ FROM\ taxi"}. This is a reality that we must face;
we do not always get to work with supercomputers that can load
everything in memory.

    \subsection{Data Overview}\label{data-overview}

Below is the schema for the \texttt{taxi} database:

\begin{verbatim}
CREATE TABLE taxi_train(
  "record_id" integer primary key,
  "VendorID" INTEGER,
  "tpep_pickup_datetime" TEXT,
  "tpep_dropoff_datetime" TEXT,
  "passenger_count" INTEGER,
  "trip_distance" REAL,
  "pickup_longitude" REAL,
  "pickup_latitude" REAL,
  "RatecodeID" INTEGER,
  "store_and_fwd_flag" TEXT,
  "dropoff_longitude" REAL,
  "dropoff_latitude" REAL,
  "payment_type" INTEGER,
  "fare_amount" REAL,
  "extra" REAL,
  "mta_tax" REAL,
  "tip_amount" REAL,
  "tolls_amount" REAL,
  "improvement_surcharge" REAL,
  "total_amount" REAL
);   
\end{verbatim}

Here is a description for your convenience: - \texttt{recordID}: primary
key of this dataset - \texttt{VendorID}: a code indicating the provider
associated with the trip record - \texttt{passenger\_count}: the number
of passengers in the vehicle (driver entered value) -
\texttt{trip\_distance}: trip distance - \texttt{dropoff\_datetime}:
date and time when the meter was engaged - \texttt{pickup\_datetime}:
date and time when the meter was disengaged -
\texttt{pickup\_longitude}: the longitude where the meter was engaged -
\texttt{pickup\_latitude}: the latitude where the meter was engaged -
\texttt{dropoff\_longitude}: the longitude where the meter was
disengaged - \texttt{dropoff\_latitude}: the latitude where the meter
was disengaged - \texttt{duration}: duration of the trip in seconds -
\texttt{payment\_type}: the payment type - \texttt{fare\_amount}: the
time-and-distance fare calculated by the meter - \texttt{extra}:
miscellaneous extras and surcharges - \texttt{mta\_tax}: MTA tax that is
automatically triggered based on the metered rate in use\\
- \texttt{tip\_amount}: the amount of credit card tips, cash tips are
not included - \texttt{tolls\_amount}: amount paid for tolls -
\texttt{improvement\_surcharge}: fixed fee - \texttt{total\_amount}:
total amount paid by passengers, cash tips are not included

    \subsection{Question 1: SQL Warmup}\label{question-1-sql-warmup}

Let's begin with some SQL questions! Remember, be careful not to select
too many entries in your query. Your kernel \textbf{will} crash! Please
write your queries in the provided triple quotes and format them with
proper SQL style. Below is an example which grabs the first 5 rows from
the \texttt{taxi} database.

We will use the \texttt{timeit} contextmanager from the \texttt{utils}
file to time each SQL execution. \emph{Beware that SQL can be slow
sometimes; enterprise SQL quries often run for hours or days!} (several
minutes execution time is
\href{https://hortonworks.com/blog/benchmarking-apache-hive-13-enterprise-hadoop/}{considered
fast}). In each cell, we have added anitipated execution time to use as
a guideline for writing your quries.

    \begin{Verbatim}[commandchars=\\\{\}]
{\color{incolor}In [{\color{incolor}6}]:} \PY{n}{q1x\PYZus{}query} \PY{o}{=} \PY{n}{f}\PY{l+s+s2}{\PYZdq{}\PYZdq{}\PYZdq{}}
        \PY{l+s+s2}{            SELECT * }
        \PY{l+s+s2}{            FROM }\PY{l+s+si}{\PYZob{}TABLE\PYZus{}NAME\PYZcb{}}
        \PY{l+s+s2}{            LIMIT 5;}
        \PY{l+s+s2}{            }\PY{l+s+s2}{\PYZdq{}\PYZdq{}\PYZdq{}}
        
        \PY{k}{with} \PY{n}{timeit}\PY{p}{(}\PY{p}{)}\PY{p}{:} \PY{c+c1}{\PYZsh{} this query should take less than a second}
            \PY{n}{q1x\PYZus{}df} \PY{o}{=} \PY{n}{pd}\PY{o}{.}\PY{n}{read\PYZus{}sql}\PY{p}{(}\PY{n}{q1x\PYZus{}query}\PY{p}{,} \PY{n}{sql\PYZus{}engine}\PY{p}{)}
        \PY{n}{q1x\PYZus{}df}\PY{o}{.}\PY{n}{head}\PY{p}{(}\PY{p}{)}
\end{Verbatim}


    \begin{Verbatim}[commandchars=\\\{\}]
0.01 s elapsed

    \end{Verbatim}

\begin{Verbatim}[commandchars=\\\{\}]
{\color{outcolor}Out[{\color{outcolor}6}]:}    record\_id  VendorID tpep\_pickup\_datetime tpep\_dropoff\_datetime  \textbackslash{}
        0          1         2  2016-01-01 00:00:00   2016-01-01 00:00:00   
        1          8         1  2016-01-01 00:00:01   2016-01-01 00:11:55   
        2         17         2  2016-01-01 00:00:05   2016-01-01 00:07:14   
        3         18         1  2016-01-01 00:00:06   2016-01-01 00:04:44   
        4         22         2  2016-01-01 00:00:08   2016-01-01 00:18:51   
        
           passenger\_count  trip\_distance  pickup\_longitude  pickup\_latitude  \textbackslash{}
        0                2           1.10        -73.990372        40.734695   
        1                1           1.20        -73.979424        40.744614   
        2                1           1.92        -73.973091        40.795361   
        3                1           1.70        -73.982101        40.774696   
        4                1           3.09        -73.999069        40.720173   
        
           RatecodeID store\_and\_fwd\_flag  dropoff\_longitude  dropoff\_latitude  \textbackslash{}
        0           1                  N         -73.981842         40.732407   
        1           1                  N         -73.992035         40.753944   
        2           1                  N         -73.978371         40.773151   
        3           1                  Y         -73.970940         40.796707   
        4           1                  N         -73.973389         40.756561   
        
           payment\_type  fare\_amount  extra  mta\_tax  tip\_amount  tolls\_amount  \textbackslash{}
        0             2          7.5    0.5      0.5        0.00           0.0   
        1             2          9.0    0.5      0.5        0.00           0.0   
        2             2          7.5    0.5      0.5        0.00           0.0   
        3             1          7.0    0.5      0.5        1.65           0.0   
        4             2         14.5    0.5      0.5        0.00           0.0   
        
           improvement\_surcharge  total\_amount  
        0                    0.3          8.80  
        1                    0.3         10.30  
        2                    0.3          8.80  
        3                    0.3          9.95  
        4                    0.3         15.80  
\end{Verbatim}
            
    \subsubsection{Question 1a}\label{question-1a}

Select the top 1000 rows from the \texttt{taxi} database ordered by
descending \texttt{total\_amount}. Note that this data is real uncleaned
data, with all the strange quirks that come from such datasets, e.g.
you'll see that the most expensive taxi ride was \$153,296.22, which is
certainly some sort of error in the data.

    \begin{Verbatim}[commandchars=\\\{\}]
{\color{incolor}In [{\color{incolor}7}]:} \PY{n}{q1a\PYZus{}query} \PY{o}{=} \PY{n}{f}\PY{l+s+s2}{\PYZdq{}\PYZdq{}\PYZdq{}}
        \PY{l+s+s2}{SELECT *}
        \PY{l+s+s2}{FROM taxi}
        \PY{l+s+s2}{ORDER BY total\PYZus{}amount DESC}
        \PY{l+s+s2}{LIMIT 1000}
        \PY{l+s+s2}{            }\PY{l+s+s2}{\PYZdq{}\PYZdq{}\PYZdq{}}
        
        \PY{c+c1}{\PYZsh{} YOUR CODE HERE}
        \PY{c+c1}{\PYZsh{}\PYZsh{}raise NotImplementedError()}
        \PY{k}{with} \PY{n}{timeit}\PY{p}{(}\PY{p}{)}\PY{p}{:} \PY{c+c1}{\PYZsh{} This query is expected to run for less than 20 seconds.}
            \PY{n}{q1a\PYZus{}df} \PY{o}{=} \PY{n}{pd}\PY{o}{.}\PY{n}{read\PYZus{}sql}\PY{p}{(}\PY{n}{q1a\PYZus{}query}\PY{p}{,} \PY{n}{sql\PYZus{}engine}\PY{p}{)}
        \PY{n}{q1a\PYZus{}df}\PY{o}{.}\PY{n}{head}\PY{p}{(}\PY{p}{)}
\end{Verbatim}


    \begin{Verbatim}[commandchars=\\\{\}]
17.61 s elapsed

    \end{Verbatim}

\begin{Verbatim}[commandchars=\\\{\}]
{\color{outcolor}Out[{\color{outcolor}7}]:}    record\_id  VendorID tpep\_pickup\_datetime tpep\_dropoff\_datetime  \textbackslash{}
        0   15958593         1  2016-02-16 18:20:33   2016-02-16 18:36:33   
        1   28810418         1  2016-03-20 11:44:34   2016-03-20 12:03:29   
        2   63007353         1  2016-06-13 15:06:32   2016-06-13 15:07:36   
        3   58271050         2  2016-05-27 14:38:36   2016-05-27 15:10:15   
        4   50682006         1  2016-05-11 22:26:52   2016-05-11 22:32:08   
        
           passenger\_count  trip\_distance  pickup\_longitude  pickup\_latitude  \textbackslash{}
        0                1       151694.0        -74.015488        40.715931   
        1                2       131091.4        -73.940979        40.819290   
        2                1            0.0        -73.980293        40.755402   
        3                1            0.0          0.000000         0.000000   
        4                1            1.8          0.000000         0.000000   
        
           RatecodeID store\_and\_fwd\_flag  dropoff\_longitude  dropoff\_latitude  \textbackslash{}
        0           1                  Y         -73.992027         40.730694   
        1           2                  N         -73.970802         40.751713   
        2           2                  N         -73.980103         40.755180   
        3           1                  N           0.000000          0.000000   
        4           1                  N           0.000000          0.000000   
        
           payment\_type  fare\_amount  extra  mta\_tax  tip\_amount  tolls\_amount  \textbackslash{}
        0             2    153231.93  63.99     0.00         0.0          0.00   
        1             3    126348.88  16.38     1.32         0.0          0.00   
        2             3      8452.00   0.00     0.50         0.0          1.44   
        3             2      4886.00   0.50     0.50         0.0          0.00   
        4             3      3006.00 -34.72    35.52         0.0          0.00   
        
           improvement\_surcharge  total\_amount  
        0                    0.3     153296.22  
        1                    0.0     126366.58  
        2                    0.3       8454.24  
        3                    0.3       4887.30  
        4                    0.0       3006.80  
\end{Verbatim}
            
    \begin{Verbatim}[commandchars=\\\{\}]
{\color{incolor}In [{\color{incolor}8}]:} \PY{k}{assert} \PY{n+nb}{len}\PY{p}{(}\PY{n}{q1a\PYZus{}df}\PY{p}{)} \PY{o}{==} \PY{l+m+mi}{1000}
        \PY{k}{assert} \PY{n}{q1a\PYZus{}df}\PY{o}{.}\PY{n}{loc}\PY{p}{[}\PY{l+m+mi}{0}\PY{p}{,} \PY{l+s+s1}{\PYZsq{}}\PY{l+s+s1}{total\PYZus{}amount}\PY{l+s+s1}{\PYZsq{}}\PY{p}{]} \PY{o}{\PYZgt{}}\PY{o}{=} \PY{n}{q1a\PYZus{}df}\PY{o}{.}\PY{n}{loc}\PY{p}{[}\PY{l+m+mi}{999}\PY{p}{,} \PY{l+s+s2}{\PYZdq{}}\PY{l+s+s2}{total\PYZus{}amount}\PY{l+s+s2}{\PYZdq{}}\PY{p}{]}
\end{Verbatim}


    \subsubsection{Question 1b}\label{question-1b}

Get the \texttt{mean}, \texttt{max} and \texttt{min}
\texttt{total\_amount} for each vendor. As above, you'll get strange
answers, since finding the min and max of a big uncleaned dataset
captures the most extreme outliers. Make sure your query outputs the
columns in this exact order.

    CREATE INDEX VendorID ON taxi (VendorID)

    \begin{Verbatim}[commandchars=\\\{\}]
{\color{incolor}In [{\color{incolor}9}]:} \PY{n}{q1b\PYZus{}query} \PY{o}{=} \PY{n}{f}\PY{l+s+s2}{\PYZdq{}\PYZdq{}\PYZdq{}}
        \PY{l+s+s2}{SELECT AVG(total\PYZus{}amount) AS mean, MAX(total\PYZus{}amount) AS max, MIN(total\PYZus{}amount) AS min}
        \PY{l+s+s2}{FROM taxi}
        \PY{l+s+s2}{GROUP BY VendorID}
        \PY{l+s+s2}{            }\PY{l+s+s2}{\PYZdq{}\PYZdq{}\PYZdq{}}
        
        \PY{c+c1}{\PYZsh{} YOUR CODE HERE}
        \PY{c+c1}{\PYZsh{}\PYZsh{}raise NotImplementedError()}
        \PY{k}{with} \PY{n}{timeit}\PY{p}{(}\PY{p}{)}\PY{p}{:} \PY{c+c1}{\PYZsh{} This query is expected to run for about 10 seconds.}
            \PY{n}{q1b\PYZus{}df} \PY{o}{=} \PY{n}{pd}\PY{o}{.}\PY{n}{read\PYZus{}sql\PYZus{}query}\PY{p}{(}\PY{n}{q1b\PYZus{}query}\PY{p}{,} \PY{n}{sql\PYZus{}engine}\PY{p}{)}
        \PY{n}{q1b\PYZus{}df}\PY{o}{.}\PY{n}{head}\PY{p}{(}\PY{p}{)}
\end{Verbatim}


    \begin{Verbatim}[commandchars=\\\{\}]
9.04 s elapsed

    \end{Verbatim}

\begin{Verbatim}[commandchars=\\\{\}]
{\color{outcolor}Out[{\color{outcolor}9}]:}         mean        max    min
        0  15.981053  153296.22    0.0
        1  16.276753    4887.30 -958.4
\end{Verbatim}
            
    \begin{Verbatim}[commandchars=\\\{\}]
{\color{incolor}In [{\color{incolor}10}]:} \PY{k}{assert} \PY{n}{q1b\PYZus{}df}\PY{o}{.}\PY{n}{shape} \PY{o}{==} \PY{p}{(}\PY{l+m+mi}{2}\PY{p}{,} \PY{l+m+mi}{3}\PY{p}{)}
         \PY{k}{assert} \PY{l+m+mi}{15} \PY{o}{\PYZlt{}} \PY{n}{q1b\PYZus{}df}\PY{o}{.}\PY{n}{iloc}\PY{p}{[}\PY{l+m+mi}{0}\PY{p}{,} \PY{l+m+mi}{0}\PY{p}{]} \PY{o}{\PYZlt{}} \PY{l+m+mi}{17}
         \PY{k}{assert} \PY{n}{q1b\PYZus{}df}\PY{o}{.}\PY{n}{iloc}\PY{p}{[}\PY{l+m+mi}{1}\PY{p}{,} \PY{l+m+mi}{1}\PY{p}{]} \PY{o}{==} \PY{l+m+mf}{4887.30}
         \PY{k}{assert} \PY{n}{q1b\PYZus{}df}\PY{o}{.}\PY{n}{iloc}\PY{p}{[}\PY{l+m+mi}{1}\PY{p}{,} \PY{l+m+mi}{2}\PY{p}{]} \PY{o}{==} \PY{o}{\PYZhy{}}\PY{l+m+mf}{958.4}
\end{Verbatim}


    \subsubsection{Question 1c}\label{question-1c}

Find the total amount paid and pickup time for all rides that started
June 28th, 2016, then order the result by total amount in descending
order. Again, make sure your query outputs the columns in this exact
order.

\emph{Hint:} From the schema, note that \texttt{tpep\_pickup\_datetime}
is a text field. We're effectively looking for strings that have a start
time that comes after \texttt{2016-06-28\ 00:00:00} but before
\texttt{2016-06-29\ 00:00:00}.

    \begin{Verbatim}[commandchars=\\\{\}]
{\color{incolor}In [{\color{incolor}11}]:} \PY{n}{q1c\PYZus{}query} \PY{o}{=} \PY{n}{f}\PY{l+s+s2}{\PYZdq{}\PYZdq{}\PYZdq{}}
         \PY{l+s+s2}{SELECT total\PYZus{}amount, tpep\PYZus{}pickup\PYZus{}datetime}
         \PY{l+s+s2}{FROM taxi T}
         \PY{l+s+s2}{WHERE T.tpep\PYZus{}pickup\PYZus{}datetime \PYZgt{} }\PY{l+s+s2}{\PYZsq{}}\PY{l+s+s2}{2016\PYZhy{}06\PYZhy{}28}\PY{l+s+s2}{\PYZsq{}}\PY{l+s+s2}{ AND T.tpep\PYZus{}pickup\PYZus{}datetime \PYZlt{} }\PY{l+s+s2}{\PYZsq{}}\PY{l+s+s2}{2016\PYZhy{}06\PYZhy{}29}\PY{l+s+s2}{\PYZsq{}}
         \PY{l+s+s2}{ORDER BY total\PYZus{}amount DESC}
         \PY{l+s+s2}{            }\PY{l+s+s2}{\PYZdq{}\PYZdq{}\PYZdq{}}
         
         \PY{c+c1}{\PYZsh{} YOUR CODE HERE}
         \PY{c+c1}{\PYZsh{}\PYZsh{}raise NotImplementedError()}
         \PY{k}{with} \PY{n}{timeit}\PY{p}{(}\PY{p}{)}\PY{p}{:} \PY{c+c1}{\PYZsh{} This query should take about 3 seconds.}
             \PY{n}{q1c\PYZus{}df} \PY{o}{=} \PY{n}{pd}\PY{o}{.}\PY{n}{read\PYZus{}sql\PYZus{}query}\PY{p}{(}\PY{n}{q1c\PYZus{}query}\PY{p}{,} \PY{n}{sql\PYZus{}engine}\PY{p}{)}
         \PY{n}{q1c\PYZus{}df}\PY{o}{.}\PY{n}{head}\PY{p}{(}\PY{p}{)}
\end{Verbatim}


    \begin{Verbatim}[commandchars=\\\{\}]
2.06 s elapsed

    \end{Verbatim}

\begin{Verbatim}[commandchars=\\\{\}]
{\color{outcolor}Out[{\color{outcolor}11}]:}    total\_amount tpep\_pickup\_datetime
         0        390.99  2016-06-28 12:23:13
         1        289.12  2016-06-28 15:14:42
         2        286.30  2016-06-28 00:01:13
         3        285.80  2016-06-28 13:34:12
         4        275.30  2016-06-28 21:38:13
\end{Verbatim}
            
    \begin{Verbatim}[commandchars=\\\{\}]
{\color{incolor}In [{\color{incolor}12}]:} \PY{k}{assert} \PY{n}{q1c\PYZus{}df}\PY{o}{.}\PY{n}{iloc}\PY{p}{[}\PY{l+m+mi}{0}\PY{p}{,} \PY{l+m+mi}{0}\PY{p}{]} \PY{o}{==} \PY{l+m+mf}{390.99}
         \PY{k}{assert} \PY{n}{q1c\PYZus{}df}\PY{o}{.}\PY{n}{shape} \PY{o}{==} \PY{p}{(}\PY{l+m+mi}{74857}\PY{p}{,} \PY{l+m+mi}{2}\PY{p}{)}
\end{Verbatim}


    \subsection{Question 1d}\label{question-1d}

Find all rides starting in the month of January in the year 2016,
selecting only those entries whose \texttt{record\_id} ends in 00.

Note: The rest of our questions in Part 1, Part 2 and Part 3 will be
based off of the results of this query. In part 4, you will be to use
anything else in the database for fitting a model (more later). Because
of its importance for the rest of the assignment, your query must be
correct for this question.

    \begin{Verbatim}[commandchars=\\\{\}]
{\color{incolor}In [{\color{incolor}13}]:} \PY{n}{q1d\PYZus{}query} \PY{o}{=} \PY{n}{f}\PY{l+s+s2}{\PYZdq{}\PYZdq{}\PYZdq{}}
         \PY{l+s+s2}{SELECT * }
         \PY{l+s+s2}{            FROM }\PY{l+s+si}{\PYZob{}TABLE\PYZus{}NAME\PYZcb{}}
         \PY{l+s+s2}{            WHERE tpep\PYZus{}pickup\PYZus{}datetime}
         \PY{l+s+s2}{                BETWEEN }\PY{l+s+s2}{\PYZsq{}}\PY{l+s+s2}{2016\PYZhy{}01\PYZhy{}01}\PY{l+s+s2}{\PYZsq{}}\PY{l+s+s2}{ AND }\PY{l+s+s2}{\PYZsq{}}\PY{l+s+s2}{2016\PYZhy{}02\PYZhy{}01}\PY{l+s+s2}{\PYZsq{}}
         \PY{l+s+s2}{                AND record\PYZus{}id }\PY{l+s+s2}{\PYZpc{}}\PY{l+s+s2}{ 100 == 0}
         \PY{l+s+s2}{            ORDER BY tpep\PYZus{}pickup\PYZus{}datetime}
         \PY{l+s+s2}{            }\PY{l+s+s2}{\PYZdq{}\PYZdq{}\PYZdq{}}
         
         \PY{c+c1}{\PYZsh{} YOUR CODE HERE}
         \PY{c+c1}{\PYZsh{}\PYZsh{}raise NotImplementedError()}
         \PY{k}{with} \PY{n}{timeit}\PY{p}{(}\PY{p}{)}\PY{p}{:} \PY{c+c1}{\PYZsh{} This query should take less than 3 second}
             \PY{n}{q1d\PYZus{}df} \PY{o}{=} \PY{n}{pd}\PY{o}{.}\PY{n}{read\PYZus{}sql\PYZus{}query}\PY{p}{(}\PY{n}{q1d\PYZus{}query}\PY{p}{,} \PY{n}{sql\PYZus{}engine}\PY{p}{)}
         \PY{n}{q1d\PYZus{}df}\PY{o}{.}\PY{n}{head}\PY{p}{(}\PY{p}{)}
\end{Verbatim}


    \begin{Verbatim}[commandchars=\\\{\}]
2.44 s elapsed

    \end{Verbatim}

\begin{Verbatim}[commandchars=\\\{\}]
{\color{outcolor}Out[{\color{outcolor}13}]:}    record\_id  VendorID tpep\_pickup\_datetime tpep\_dropoff\_datetime  \textbackslash{}
         0      37300         1  2016-01-01 00:02:20   2016-01-01 00:11:58   
         1      37400         1  2016-01-01 00:03:04   2016-01-01 00:28:54   
         2      37500         2  2016-01-01 00:03:40   2016-01-01 00:12:47   
         3      37900         2  2016-01-01 00:05:38   2016-01-01 00:10:02   
         4      38500         1  2016-01-01 00:07:50   2016-01-01 00:23:42   
         
            passenger\_count  trip\_distance  pickup\_longitude  pickup\_latitude  \textbackslash{}
         0                2           1.20        -73.990578        40.732883   
         1                1           5.00        -73.994286        40.749153   
         2                6           2.54        -73.949821        40.785412   
         3                3           0.76        -74.002998        40.739220   
         4                1           2.40        -73.992546        40.766624   
         
            RatecodeID store\_and\_fwd\_flag  dropoff\_longitude  dropoff\_latitude  \textbackslash{}
         0           1                  N         -73.982307         40.747406   
         1           1                  N         -73.956688         40.747395   
         2           1                  N         -73.974586         40.758282   
         3           1                  N         -74.006714         40.744259   
         4           1                  N         -73.958771         40.763844   
         
            payment\_type  fare\_amount  extra  mta\_tax  tip\_amount  tolls\_amount  \textbackslash{}
         0             2          8.0    0.5      0.5        0.00           0.0   
         1             1         20.5    0.5      0.5        1.09           0.0   
         2             1          9.5    0.5      0.5        2.16           0.0   
         3             2          5.0    0.5      0.5        0.00           0.0   
         4             1         12.0    0.5      0.5        2.00           0.0   
         
            improvement\_surcharge  total\_amount  
         0                    0.3          9.30  
         1                    0.3         22.89  
         2                    0.3         12.96  
         3                    0.3          6.30  
         4                    0.3         15.30  
\end{Verbatim}
            
    \begin{Verbatim}[commandchars=\\\{\}]
{\color{incolor}In [{\color{incolor}14}]:} \PY{k}{assert} \PY{n}{q1d\PYZus{}df}\PY{o}{.}\PY{n}{iloc}\PY{p}{[}\PY{l+m+mi}{0}\PY{p}{]}\PY{o}{.}\PY{n}{loc}\PY{p}{[}\PY{l+s+s1}{\PYZsq{}}\PY{l+s+s1}{tpep\PYZus{}pickup\PYZus{}datetime}\PY{l+s+s1}{\PYZsq{}}\PY{p}{]} \PY{o}{\PYZgt{}}\PY{o}{=} \PY{l+s+s2}{\PYZdq{}}\PY{l+s+s2}{2016\PYZhy{}01\PYZhy{}01}\PY{l+s+s2}{\PYZdq{}}
         \PY{k}{assert} \PY{n}{q1d\PYZus{}df}\PY{o}{.}\PY{n}{iloc}\PY{p}{[}\PY{o}{\PYZhy{}}\PY{l+m+mi}{1}\PY{p}{]}\PY{o}{.}\PY{n}{loc}\PY{p}{[}\PY{l+s+s1}{\PYZsq{}}\PY{l+s+s1}{tpep\PYZus{}pickup\PYZus{}datetime}\PY{l+s+s1}{\PYZsq{}}\PY{p}{]} \PY{o}{\PYZlt{}}\PY{o}{=} \PY{l+s+s2}{\PYZdq{}}\PY{l+s+s2}{2016\PYZhy{}02\PYZhy{}01}\PY{l+s+s2}{\PYZdq{}}
         \PY{k}{assert} \PY{n}{q1d\PYZus{}df}\PY{o}{.}\PY{n}{shape} \PY{o}{==} \PY{p}{(}\PY{l+m+mi}{23674}\PY{p}{,} \PY{l+m+mi}{20}\PY{p}{)}
\end{Verbatim}


    \subsection{Question 2: Data
Inspection}\label{question-2-data-inspection}

We will refer to the table generated by Question 1d as \texttt{Jan16}.
Note that we have not explicitly built a table called \texttt{Jan16} in
our SQL database. We are instead using \texttt{Jan16} to represent the
mathematical object that results from Question 1d. Let us now check some
basic properties of \texttt{Jan16}. We will be addressing the following
properties within our dataset: - missing data values - duplicated values
- range of duration values - range of latitude and longitude values -
range of passenger count values

It is good practice to check these properties when presented with a new
dataset. There are two ways to check these properties: Approach one is
to write SQL queries that directly interact with the database. Approach
two is to create a pandas dataframe and use pandas methods. Since you've
already gotten similar practice with pandas earlier in the semester,
we'll stick with approach one.

In the following problems, you'll check these properties using SQL
queries. We'll also provide you with the pandas solution so that you can
compare with your SQL based solution. In order to be able to provide
these pandas solutions, we need to store the result of your
\texttt{q1d\_query} into a dataframe, which we'll call
\texttt{jan\_16\_df}.

    \begin{Verbatim}[commandchars=\\\{\}]
{\color{incolor}In [{\color{incolor}15}]:} \PY{k}{with} \PY{n}{timeit}\PY{p}{(}\PY{p}{)}\PY{p}{:} \PY{c+c1}{\PYZsh{} less than 3 seconds}
             \PY{n}{jan\PYZus{}16\PYZus{}df} \PY{o}{=} \PY{n}{pd}\PY{o}{.}\PY{n}{read\PYZus{}sql\PYZus{}query}\PY{p}{(}\PY{n}{q1d\PYZus{}query}\PY{p}{,} \PY{n}{sql\PYZus{}engine}\PY{p}{)}
         \PY{n}{jan\PYZus{}16\PYZus{}df}\PY{p}{[}\PY{l+s+s1}{\PYZsq{}}\PY{l+s+s1}{tpep\PYZus{}pickup\PYZus{}datetime}\PY{l+s+s1}{\PYZsq{}}\PY{p}{]} \PY{o}{=} \PY{n}{pd}\PY{o}{.}\PY{n}{to\PYZus{}datetime}\PY{p}{(}\PY{n}{jan\PYZus{}16\PYZus{}df}\PY{p}{[}\PY{l+s+s1}{\PYZsq{}}\PY{l+s+s1}{tpep\PYZus{}pickup\PYZus{}datetime}\PY{l+s+s1}{\PYZsq{}}\PY{p}{]}\PY{p}{)}
         \PY{n}{jan\PYZus{}16\PYZus{}df}\PY{p}{[}\PY{l+s+s1}{\PYZsq{}}\PY{l+s+s1}{tpep\PYZus{}dropoff\PYZus{}datetime}\PY{l+s+s1}{\PYZsq{}}\PY{p}{]} \PY{o}{=} \PY{n}{pd}\PY{o}{.}\PY{n}{to\PYZus{}datetime}\PY{p}{(}\PY{n}{jan\PYZus{}16\PYZus{}df}\PY{p}{[}\PY{l+s+s1}{\PYZsq{}}\PY{l+s+s1}{tpep\PYZus{}dropoff\PYZus{}datetime}\PY{l+s+s1}{\PYZsq{}}\PY{p}{]}\PY{p}{)}
         \PY{n}{jan\PYZus{}16\PYZus{}df}\PY{o}{.}\PY{n}{head}\PY{p}{(}\PY{p}{)}
\end{Verbatim}


    \begin{Verbatim}[commandchars=\\\{\}]
2.43 s elapsed

    \end{Verbatim}

\begin{Verbatim}[commandchars=\\\{\}]
{\color{outcolor}Out[{\color{outcolor}15}]:}    record\_id  VendorID tpep\_pickup\_datetime tpep\_dropoff\_datetime  \textbackslash{}
         0      37300         1  2016-01-01 00:02:20   2016-01-01 00:11:58   
         1      37400         1  2016-01-01 00:03:04   2016-01-01 00:28:54   
         2      37500         2  2016-01-01 00:03:40   2016-01-01 00:12:47   
         3      37900         2  2016-01-01 00:05:38   2016-01-01 00:10:02   
         4      38500         1  2016-01-01 00:07:50   2016-01-01 00:23:42   
         
            passenger\_count  trip\_distance  pickup\_longitude  pickup\_latitude  \textbackslash{}
         0                2           1.20        -73.990578        40.732883   
         1                1           5.00        -73.994286        40.749153   
         2                6           2.54        -73.949821        40.785412   
         3                3           0.76        -74.002998        40.739220   
         4                1           2.40        -73.992546        40.766624   
         
            RatecodeID store\_and\_fwd\_flag  dropoff\_longitude  dropoff\_latitude  \textbackslash{}
         0           1                  N         -73.982307         40.747406   
         1           1                  N         -73.956688         40.747395   
         2           1                  N         -73.974586         40.758282   
         3           1                  N         -74.006714         40.744259   
         4           1                  N         -73.958771         40.763844   
         
            payment\_type  fare\_amount  extra  mta\_tax  tip\_amount  tolls\_amount  \textbackslash{}
         0             2          8.0    0.5      0.5        0.00           0.0   
         1             1         20.5    0.5      0.5        1.09           0.0   
         2             1          9.5    0.5      0.5        2.16           0.0   
         3             2          5.0    0.5      0.5        0.00           0.0   
         4             1         12.0    0.5      0.5        2.00           0.0   
         
            improvement\_surcharge  total\_amount  
         0                    0.3          9.30  
         1                    0.3         22.89  
         2                    0.3         12.96  
         3                    0.3          6.30  
         4                    0.3         15.30  
\end{Verbatim}
            
    For the remaining questions in part 1, you'll be using nested queries.
For example, the nested query below selects all rides with passenger
count equal to 2 from \texttt{Jan16}. Reminder that Python automatically
replaces the \texttt{"q1d\_query"} in
\texttt{temporary\_table\_query\_example} with the contents of the
string variable named \texttt{q1d\_query}.

    \begin{Verbatim}[commandchars=\\\{\}]
{\color{incolor}In [{\color{incolor}16}]:} \PY{c+c1}{\PYZsh{} Jan16 to dataframe using temporary table}
         \PY{n}{temporary\PYZus{}table\PYZus{}query\PYZus{}example} \PY{o}{=} \PY{n}{f}\PY{l+s+s2}{\PYZdq{}\PYZdq{}\PYZdq{}}
         \PY{l+s+s2}{SELECT *}
         \PY{l+s+s2}{FROM (}\PY{l+s+si}{\PYZob{}q1d\PYZus{}query\PYZcb{}}\PY{l+s+s2}{)}
         \PY{l+s+s2}{WHERE passenger\PYZus{}count = 2;}\PY{l+s+s2}{\PYZdq{}\PYZdq{}\PYZdq{}}
         \PY{n+nb}{print}\PY{p}{(}\PY{n}{temporary\PYZus{}table\PYZus{}query\PYZus{}example}\PY{p}{)}
\end{Verbatim}


    \begin{Verbatim}[commandchars=\\\{\}]

SELECT *
FROM (
SELECT * 
            FROM taxi
            WHERE tpep\_pickup\_datetime
                BETWEEN '2016-01-01' AND '2016-02-01'
                AND record\_id \% 100 == 0
            ORDER BY tpep\_pickup\_datetime
            )
WHERE passenger\_count = 2;

    \end{Verbatim}

    The cell below executes this nested query.

    \begin{Verbatim}[commandchars=\\\{\}]
{\color{incolor}In [{\color{incolor}17}]:} \PY{k}{with} \PY{n}{timeit}\PY{p}{(}\PY{p}{)}\PY{p}{:} \PY{c+c1}{\PYZsh{} less than 3 seconds}
             \PY{n}{pd}\PY{o}{.}\PY{n}{read\PYZus{}sql\PYZus{}query}\PY{p}{(}\PY{n}{temporary\PYZus{}table\PYZus{}query\PYZus{}example}\PY{p}{,} \PY{n}{sql\PYZus{}engine}\PY{p}{)}
\end{Verbatim}


    \begin{Verbatim}[commandchars=\\\{\}]
2.28 s elapsed

    \end{Verbatim}

    \paragraph{Question 2a}\label{question-2a}

Write a SQL query to check if \texttt{Jan16} contains any missing
values. Unfortunately, in this table, missing values are \emph{not}
specified with NaN nor empty strings. For example, take a look at record
ID 136700. What do you observe about the location information?

Write a SQL query \texttt{q2a\_query} that collects all rows that have a
missing \texttt{tpep\_pickup\_datetime},
\texttt{tpep\_dropoff\_datetime}, \texttt{pickup\_longitude}, or
\texttt{pickup\_latitude}. Then set
\texttt{number\_of\_rows\_with\_missing\_values} to the number of rows
that have at least one missing value.

In pandas, we could use boolean indexing to filter out these values.

    \begin{Verbatim}[commandchars=\\\{\}]
{\color{incolor}In [{\color{incolor}18}]:} \PY{c+c1}{\PYZsh{} Inspecting record 136700 for your convience.}
         
         \PY{n}{pd}\PY{o}{.}\PY{n}{read\PYZus{}sql\PYZus{}query}\PY{p}{(}\PY{n}{f}\PY{l+s+s2}{\PYZdq{}\PYZdq{}\PYZdq{}}
         \PY{l+s+s2}{SELECT * }
         \PY{l+s+s2}{FROM }\PY{l+s+si}{\PYZob{}TABLE\PYZus{}NAME\PYZcb{}}\PY{l+s+s2}{ }
         \PY{l+s+s2}{WHERE record\PYZus{}id = 136700}
         \PY{l+s+s2}{\PYZdq{}\PYZdq{}\PYZdq{}}\PY{p}{,} \PY{n}{sql\PYZus{}engine}\PY{p}{)}
\end{Verbatim}


\begin{Verbatim}[commandchars=\\\{\}]
{\color{outcolor}Out[{\color{outcolor}18}]:}    record\_id  VendorID tpep\_pickup\_datetime tpep\_dropoff\_datetime  \textbackslash{}
         0     136700         1  2016-01-01 03:13:07   2016-01-01 03:28:48   
         
            passenger\_count  trip\_distance  pickup\_longitude  pickup\_latitude  \textbackslash{}
         0                1            3.3               0.0              0.0   
         
            RatecodeID store\_and\_fwd\_flag  dropoff\_longitude  dropoff\_latitude  \textbackslash{}
         0           1                  N         -73.978462         40.783058   
         
            payment\_type  fare\_amount  extra  mta\_tax  tip\_amount  tolls\_amount  \textbackslash{}
         0             1         13.5    0.5      0.5        2.95           0.0   
         
            improvement\_surcharge  total\_amount  
         0                    0.3         17.75  
\end{Verbatim}
            
    \begin{Verbatim}[commandchars=\\\{\}]
{\color{incolor}In [{\color{incolor}19}]:} \PY{n}{q2a\PYZus{}query} \PY{o}{=} \PY{n}{f}\PY{l+s+s2}{\PYZdq{}\PYZdq{}\PYZdq{}}\PY{l+s+s2}{ }
         \PY{l+s+s2}{SELECT *}
         \PY{l+s+s2}{FROM (}\PY{l+s+si}{\PYZob{}q1d\PYZus{}query\PYZcb{}}\PY{l+s+s2}{)}
         \PY{l+s+s2}{WHERE pickup\PYZus{}longitude = 0 OR pickup\PYZus{}latitude = 0;}
         \PY{l+s+s2}{            }\PY{l+s+s2}{\PYZdq{}\PYZdq{}\PYZdq{}}
         \PY{n}{number\PYZus{}of\PYZus{}rows\PYZus{}with\PYZus{}missing\PYZus{}values} \PY{o}{=} \PY{n}{jan\PYZus{}16\PYZus{}df}\PY{p}{[}\PY{p}{(}\PY{n}{jan\PYZus{}16\PYZus{}df}\PY{p}{[}\PY{l+s+s1}{\PYZsq{}}\PY{l+s+s1}{pickup\PYZus{}longitude}\PY{l+s+s1}{\PYZsq{}}\PY{p}{]} \PY{o}{==} \PY{l+m+mi}{0}\PY{p}{)} \PY{o}{|}
                                                       \PY{p}{(}\PY{n}{jan\PYZus{}16\PYZus{}df}\PY{p}{[}\PY{l+s+s1}{\PYZsq{}}\PY{l+s+s1}{pickup\PYZus{}latitude}\PY{l+s+s1}{\PYZsq{}}\PY{p}{]} \PY{o}{==} \PY{l+m+mi}{0}\PY{p}{)}\PY{p}{]}\PY{o}{.}\PY{n}{shape}\PY{p}{[}\PY{l+m+mi}{0}\PY{p}{]}
         
         \PY{c+c1}{\PYZsh{} YOUR CODE HERE}
         \PY{c+c1}{\PYZsh{}\PYZsh{}raise NotImplementedError()}
         
         \PY{k}{with} \PY{n}{timeit}\PY{p}{(}\PY{p}{)}\PY{p}{:} \PY{c+c1}{\PYZsh{} Should take \PYZlt{} 3 seconds}
             \PY{n}{q2a\PYZus{}df} \PY{o}{=} \PY{n}{pd}\PY{o}{.}\PY{n}{read\PYZus{}sql\PYZus{}query}\PY{p}{(}\PY{n}{q2a\PYZus{}query}\PY{p}{,} \PY{n}{sql\PYZus{}engine}\PY{p}{)}
         \PY{n}{q2a\PYZus{}df}\PY{o}{.}\PY{n}{head}\PY{p}{(}\PY{p}{)}
\end{Verbatim}


    \begin{Verbatim}[commandchars=\\\{\}]
2.26 s elapsed

    \end{Verbatim}

\begin{Verbatim}[commandchars=\\\{\}]
{\color{outcolor}Out[{\color{outcolor}19}]:}    record\_id  VendorID tpep\_pickup\_datetime tpep\_dropoff\_datetime  \textbackslash{}
         0     136700         1  2016-01-01 03:13:07   2016-01-01 03:28:48   
         1     216000         1  2016-01-01 11:46:23   2016-01-01 11:57:50   
         2     228400         2  2016-01-01 12:40:12   2016-01-01 12:46:35   
         3     340400         2  2016-01-01 19:37:19   2016-01-01 20:17:57   
         4     360600         1  2016-01-01 21:06:29   2016-01-01 21:09:41   
         
            passenger\_count  trip\_distance  pickup\_longitude  pickup\_latitude  \textbackslash{}
         0                1           3.30               0.0              0.0   
         1                4           3.50               0.0              0.0   
         2                1           1.11               0.0              0.0   
         3                5          20.86               0.0              0.0   
         4                1           0.70               0.0              0.0   
         
            RatecodeID store\_and\_fwd\_flag  dropoff\_longitude  dropoff\_latitude  \textbackslash{}
         0           1                  N         -73.978462         40.783058   
         1           1                  N           0.000000          0.000000   
         2           1                  N           0.000000          0.000000   
         3           2                  N           0.000000          0.000000   
         4           1                  N           0.000000          0.000000   
         
            payment\_type  fare\_amount  extra  mta\_tax  tip\_amount  tolls\_amount  \textbackslash{}
         0             1         13.5    0.5      0.5        2.95           0.0   
         1             1         12.5    0.0      0.5        2.00           0.0   
         2             2          6.5    0.0      0.5        0.00           0.0   
         3             1         52.0    0.0      0.5       10.56           0.0   
         4             2          4.5    0.5      0.5        0.00           0.0   
         
            improvement\_surcharge  total\_amount  
         0                    0.3         17.75  
         1                    0.3         15.30  
         2                    0.3          7.30  
         3                    0.3         63.36  
         4                    0.3          5.80  
\end{Verbatim}
            
    \begin{Verbatim}[commandchars=\\\{\}]
{\color{incolor}In [{\color{incolor}20}]:} \PY{n}{number\PYZus{}of\PYZus{}rows\PYZus{}with\PYZus{}missing\PYZus{}values}
\end{Verbatim}


\begin{Verbatim}[commandchars=\\\{\}]
{\color{outcolor}Out[{\color{outcolor}20}]:} 371
\end{Verbatim}
            
    \begin{Verbatim}[commandchars=\\\{\}]
{\color{incolor}In [{\color{incolor}21}]:} \PY{c+c1}{\PYZsh{} Hidden Test }
\end{Verbatim}


    \subsubsection{Question 2b}\label{question-2b}

Write a SQL query \texttt{q2b\_query} to help determine if there are any
duplicate records in \texttt{Jan16}. Set the boolean
\texttt{has\_duplicates} variable to \texttt{True} or \texttt{False}
based on what you learn. You may use \texttt{len(jan\_16\_df)} in your
solution.

For comparison, approach two (pandas) for duplicate checking looks like
\texttt{num\_duplicates\ =\ jan\_16\_df.duplicated(subset=jan\_16\_df.columns).sum()}.

    \begin{Verbatim}[commandchars=\\\{\}]
{\color{incolor}In [{\color{incolor}22}]:} \PY{n}{jan\PYZus{}16\PYZus{}df}\PY{o}{.}\PY{n}{head}\PY{p}{(}\PY{p}{)}
\end{Verbatim}


\begin{Verbatim}[commandchars=\\\{\}]
{\color{outcolor}Out[{\color{outcolor}22}]:}    record\_id  VendorID tpep\_pickup\_datetime tpep\_dropoff\_datetime  \textbackslash{}
         0      37300         1  2016-01-01 00:02:20   2016-01-01 00:11:58   
         1      37400         1  2016-01-01 00:03:04   2016-01-01 00:28:54   
         2      37500         2  2016-01-01 00:03:40   2016-01-01 00:12:47   
         3      37900         2  2016-01-01 00:05:38   2016-01-01 00:10:02   
         4      38500         1  2016-01-01 00:07:50   2016-01-01 00:23:42   
         
            passenger\_count  trip\_distance  pickup\_longitude  pickup\_latitude  \textbackslash{}
         0                2           1.20        -73.990578        40.732883   
         1                1           5.00        -73.994286        40.749153   
         2                6           2.54        -73.949821        40.785412   
         3                3           0.76        -74.002998        40.739220   
         4                1           2.40        -73.992546        40.766624   
         
            RatecodeID store\_and\_fwd\_flag  dropoff\_longitude  dropoff\_latitude  \textbackslash{}
         0           1                  N         -73.982307         40.747406   
         1           1                  N         -73.956688         40.747395   
         2           1                  N         -73.974586         40.758282   
         3           1                  N         -74.006714         40.744259   
         4           1                  N         -73.958771         40.763844   
         
            payment\_type  fare\_amount  extra  mta\_tax  tip\_amount  tolls\_amount  \textbackslash{}
         0             2          8.0    0.5      0.5        0.00           0.0   
         1             1         20.5    0.5      0.5        1.09           0.0   
         2             1          9.5    0.5      0.5        2.16           0.0   
         3             2          5.0    0.5      0.5        0.00           0.0   
         4             1         12.0    0.5      0.5        2.00           0.0   
         
            improvement\_surcharge  total\_amount  
         0                    0.3          9.30  
         1                    0.3         22.89  
         2                    0.3         12.96  
         3                    0.3          6.30  
         4                    0.3         15.30  
\end{Verbatim}
            
    \begin{Verbatim}[commandchars=\\\{\}]
{\color{incolor}In [{\color{incolor}23}]:} \PY{n}{num\PYZus{}duplicates} \PY{o}{=} \PY{n}{jan\PYZus{}16\PYZus{}df}\PY{o}{.}\PY{n}{duplicated}\PY{p}{(}\PY{n}{subset}\PY{o}{=}\PY{n}{jan\PYZus{}16\PYZus{}df}\PY{o}{.}\PY{n}{columns}\PY{p}{)}\PY{o}{.}\PY{n}{sum}\PY{p}{(}\PY{p}{)}
         \PY{n}{num\PYZus{}duplicates}  \PY{c+c1}{\PYZsh{}0}
\end{Verbatim}


\begin{Verbatim}[commandchars=\\\{\}]
{\color{outcolor}Out[{\color{outcolor}23}]:} 0
\end{Verbatim}
            
    \begin{Verbatim}[commandchars=\\\{\}]
{\color{incolor}In [{\color{incolor}24}]:} \PY{n}{q2b\PYZus{}query} \PY{o}{=} \PY{n}{f}\PY{l+s+s2}{\PYZdq{}\PYZdq{}\PYZdq{}}
         \PY{l+s+s2}{SELECT *}
         \PY{l+s+s2}{FROM (}\PY{l+s+si}{\PYZob{}q1d\PYZus{}query\PYZcb{}}\PY{l+s+s2}{)}
         \PY{l+s+s2}{GROUP BY record\PYZus{}id}
         \PY{l+s+s2}{HAVING count(record\PYZus{}id) \PYZgt{} 1}
         \PY{l+s+s2}{            }\PY{l+s+s2}{\PYZdq{}\PYZdq{}\PYZdq{}}
         
         \PY{n}{has\PYZus{}duplicates} \PY{o}{=} \PY{k+kc}{False} \PY{c+c1}{\PYZsh{} True or False}
         
         \PY{c+c1}{\PYZsh{} YOUR CODE HERE}
         \PY{c+c1}{\PYZsh{}\PYZsh{}raise NotImplementedError()}
         
         \PY{k}{with} \PY{n}{timeit}\PY{p}{(}\PY{p}{)}\PY{p}{:} \PY{c+c1}{\PYZsh{} should take \PYZlt{} 3 seconds}
             \PY{n}{q2b\PYZus{}df} \PY{o}{=} \PY{n}{pd}\PY{o}{.}\PY{n}{read\PYZus{}sql\PYZus{}query}\PY{p}{(}\PY{n}{q2b\PYZus{}query}\PY{p}{,} \PY{n}{sql\PYZus{}engine}\PY{p}{)}
         \PY{n}{q2b\PYZus{}df}\PY{o}{.}\PY{n}{head}\PY{p}{(}\PY{p}{)}
\end{Verbatim}


    \begin{Verbatim}[commandchars=\\\{\}]
2.40 s elapsed

    \end{Verbatim}

\begin{Verbatim}[commandchars=\\\{\}]
{\color{outcolor}Out[{\color{outcolor}24}]:} Empty DataFrame
         Columns: [record\_id, VendorID, tpep\_pickup\_datetime, tpep\_dropoff\_datetime, passenger\_count, trip\_distance, pickup\_longitude, pickup\_latitude, RatecodeID, store\_and\_fwd\_flag, dropoff\_longitude, dropoff\_latitude, payment\_type, fare\_amount, extra, mta\_tax, tip\_amount, tolls\_amount, improvement\_surcharge, total\_amount]
         Index: []
\end{Verbatim}
            
    \begin{Verbatim}[commandchars=\\\{\}]
{\color{incolor}In [{\color{incolor}25}]:} \PY{c+c1}{\PYZsh{} Hidden test}
\end{Verbatim}


    \subsubsection{Question 2c}\label{question-2c}

Find the min and max trip duration in \texttt{Jan16}. You may manually
fill in the \texttt{min\_duration}, \texttt{max\_duration} placeholders.

\emph{Hint:} check
\href{https://www.techonthenet.com/sqlite/functions/julianday.php}{\texttt{julianday}}
in \texttt{SQLite}. Your answer should be decimal representations of a
day (e.g. 6 hours = 0.25).

    \begin{Verbatim}[commandchars=\\\{\}]
{\color{incolor}In [{\color{incolor}26}]:} \PY{n}{jan\PYZus{}16\PYZus{}df}\PY{o}{.}\PY{n}{head}\PY{p}{(}\PY{p}{)}
\end{Verbatim}


\begin{Verbatim}[commandchars=\\\{\}]
{\color{outcolor}Out[{\color{outcolor}26}]:}    record\_id  VendorID tpep\_pickup\_datetime tpep\_dropoff\_datetime  \textbackslash{}
         0      37300         1  2016-01-01 00:02:20   2016-01-01 00:11:58   
         1      37400         1  2016-01-01 00:03:04   2016-01-01 00:28:54   
         2      37500         2  2016-01-01 00:03:40   2016-01-01 00:12:47   
         3      37900         2  2016-01-01 00:05:38   2016-01-01 00:10:02   
         4      38500         1  2016-01-01 00:07:50   2016-01-01 00:23:42   
         
            passenger\_count  trip\_distance  pickup\_longitude  pickup\_latitude  \textbackslash{}
         0                2           1.20        -73.990578        40.732883   
         1                1           5.00        -73.994286        40.749153   
         2                6           2.54        -73.949821        40.785412   
         3                3           0.76        -74.002998        40.739220   
         4                1           2.40        -73.992546        40.766624   
         
            RatecodeID store\_and\_fwd\_flag  dropoff\_longitude  dropoff\_latitude  \textbackslash{}
         0           1                  N         -73.982307         40.747406   
         1           1                  N         -73.956688         40.747395   
         2           1                  N         -73.974586         40.758282   
         3           1                  N         -74.006714         40.744259   
         4           1                  N         -73.958771         40.763844   
         
            payment\_type  fare\_amount  extra  mta\_tax  tip\_amount  tolls\_amount  \textbackslash{}
         0             2          8.0    0.5      0.5        0.00           0.0   
         1             1         20.5    0.5      0.5        1.09           0.0   
         2             1          9.5    0.5      0.5        2.16           0.0   
         3             2          5.0    0.5      0.5        0.00           0.0   
         4             1         12.0    0.5      0.5        2.00           0.0   
         
            improvement\_surcharge  total\_amount  
         0                    0.3          9.30  
         1                    0.3         22.89  
         2                    0.3         12.96  
         3                    0.3          6.30  
         4                    0.3         15.30  
\end{Verbatim}
            
    \begin{Verbatim}[commandchars=\\\{\}]
{\color{incolor}In [{\color{incolor}27}]:} \PY{n}{q2c\PYZus{}query} \PY{o}{=} \PY{n}{f}\PY{l+s+s2}{\PYZdq{}\PYZdq{}\PYZdq{}}
         \PY{l+s+s2}{SELECT MAX(julianday(tpep\PYZus{}dropoff\PYZus{}datetime) \PYZhy{} julianday(tpep\PYZus{}pickup\PYZus{}datetime)) as max\PYZus{}duration,}
         \PY{l+s+s2}{MIN(julianday(tpep\PYZus{}dropoff\PYZus{}datetime) \PYZhy{} julianday(tpep\PYZus{}pickup\PYZus{}datetime)) as min\PYZus{}duration}
         \PY{l+s+s2}{FROM (}\PY{l+s+si}{\PYZob{}q1d\PYZus{}query\PYZcb{}}\PY{l+s+s2}{)}
         \PY{l+s+s2}{            }\PY{l+s+s2}{\PYZdq{}\PYZdq{}\PYZdq{}}
         
         \PY{n}{min\PYZus{}duration} \PY{o}{=} \PY{l+m+mf}{0.0}
         \PY{n}{max\PYZus{}duration} \PY{o}{=} \PY{l+m+mf}{0.99919}
         \PY{c+c1}{\PYZsh{} YOUR CODE HERE}
         \PY{c+c1}{\PYZsh{}\PYZsh{}raise NotImplementedError()}
         \PY{k}{with} \PY{n}{timeit}\PY{p}{(}\PY{p}{)}\PY{p}{:} \PY{c+c1}{\PYZsh{} should take \PYZlt{} 3 seconds}
             \PY{n}{q2c\PYZus{}df} \PY{o}{=} \PY{n}{pd}\PY{o}{.}\PY{n}{read\PYZus{}sql\PYZus{}query}\PY{p}{(}\PY{n}{q2c\PYZus{}query}\PY{p}{,} \PY{n}{sql\PYZus{}engine}\PY{p}{)}
         \PY{n}{q2c\PYZus{}df}\PY{o}{.}\PY{n}{head}\PY{p}{(}\PY{p}{)}
\end{Verbatim}


    \begin{Verbatim}[commandchars=\\\{\}]
2.28 s elapsed

    \end{Verbatim}

\begin{Verbatim}[commandchars=\\\{\}]
{\color{outcolor}Out[{\color{outcolor}27}]:}    max\_duration  min\_duration
         0       0.99919           0.0
\end{Verbatim}
            
    \begin{Verbatim}[commandchars=\\\{\}]
{\color{incolor}In [{\color{incolor}28}]:} \PY{n}{df\PYZus{}min\PYZus{}seconds} \PY{o}{=} \PY{n+nb}{min}\PY{p}{(}\PY{n}{jan\PYZus{}16\PYZus{}df}\PY{p}{[}\PY{l+s+s2}{\PYZdq{}}\PY{l+s+s2}{tpep\PYZus{}dropoff\PYZus{}datetime}\PY{l+s+s2}{\PYZdq{}}\PY{p}{]} \PY{o}{\PYZhy{}} \PY{n}{jan\PYZus{}16\PYZus{}df}\PY{p}{[}\PY{l+s+s2}{\PYZdq{}}\PY{l+s+s2}{tpep\PYZus{}pickup\PYZus{}datetime}\PY{l+s+s2}{\PYZdq{}}\PY{p}{]}\PY{p}{)}\PY{o}{.}\PY{n}{total\PYZus{}seconds}\PY{p}{(}\PY{p}{)}
         \PY{n}{df\PYZus{}max\PYZus{}seconds} \PY{o}{=} \PY{n+nb}{max}\PY{p}{(}\PY{n}{jan\PYZus{}16\PYZus{}df}\PY{p}{[}\PY{l+s+s2}{\PYZdq{}}\PY{l+s+s2}{tpep\PYZus{}dropoff\PYZus{}datetime}\PY{l+s+s2}{\PYZdq{}}\PY{p}{]} \PY{o}{\PYZhy{}} \PY{n}{jan\PYZus{}16\PYZus{}df}\PY{p}{[}\PY{l+s+s2}{\PYZdq{}}\PY{l+s+s2}{tpep\PYZus{}pickup\PYZus{}datetime}\PY{l+s+s2}{\PYZdq{}}\PY{p}{]}\PY{p}{)}\PY{o}{.}\PY{n}{total\PYZus{}seconds}\PY{p}{(}\PY{p}{)}
         \PY{k}{assert} \PY{n}{min\PYZus{}duration} \PY{o}{==} \PY{n}{df\PYZus{}min\PYZus{}seconds}\PY{o}{/}\PY{l+m+mi}{86400}
         \PY{k}{assert} \PY{n}{np}\PY{o}{.}\PY{n}{isclose}\PY{p}{(}\PY{n}{max\PYZus{}duration}\PY{p}{,}\PY{n}{df\PYZus{}max\PYZus{}seconds}\PY{o}{/}\PY{l+m+mi}{86400}\PY{p}{)}
\end{Verbatim}


    \begin{Verbatim}[commandchars=\\\{\}]
{\color{incolor}In [{\color{incolor}29}]:} \PY{n}{df\PYZus{}min\PYZus{}seconds}
\end{Verbatim}


\begin{Verbatim}[commandchars=\\\{\}]
{\color{outcolor}Out[{\color{outcolor}29}]:} 0.0
\end{Verbatim}
            
    \begin{Verbatim}[commandchars=\\\{\}]
{\color{incolor}In [{\color{incolor}30}]:} \PY{n}{df\PYZus{}max\PYZus{}seconds}
\end{Verbatim}


\begin{Verbatim}[commandchars=\\\{\}]
{\color{outcolor}Out[{\color{outcolor}30}]:} 86330.0
\end{Verbatim}
            
    The cell above should have shown that some trips are extremely long
(almost a day)! What is up with this? There may be several reasons why
we have a handful of taxi rides with abnormally high durations.

Using our domain knowledge about taxi businesses in NYC, we might
believe that taxi drivers accidentally left their meters running, which
causes high duration values to be recorded. This is a plausible
explanation. Because of this, we will only train and predict on taxi
ride data that has a duration of at most 12 hours.

    \subsection{Question 3: Data Cleaning}\label{question-3-data-cleaning}

Now let's use domain knowledge and clean up our data. You will use SQL
while we perform the equivalent operations in pandas on
\texttt{cleaned\_jan\_16\_df}.

    \begin{Verbatim}[commandchars=\\\{\}]
{\color{incolor}In [{\color{incolor}31}]:} \PY{n}{cleaned\PYZus{}jan\PYZus{}16\PYZus{}df} \PY{o}{=} \PY{n}{jan\PYZus{}16\PYZus{}df}\PY{o}{.}\PY{n}{copy}\PY{p}{(}\PY{p}{)}
\end{Verbatim}


    \begin{Verbatim}[commandchars=\\\{\}]
{\color{incolor}In [{\color{incolor}32}]:} \PY{n}{cleaned\PYZus{}jan\PYZus{}16\PYZus{}df}\PY{o}{.}\PY{n}{head}\PY{p}{(}\PY{p}{)}
\end{Verbatim}


\begin{Verbatim}[commandchars=\\\{\}]
{\color{outcolor}Out[{\color{outcolor}32}]:}    record\_id  VendorID tpep\_pickup\_datetime tpep\_dropoff\_datetime  \textbackslash{}
         0      37300         1  2016-01-01 00:02:20   2016-01-01 00:11:58   
         1      37400         1  2016-01-01 00:03:04   2016-01-01 00:28:54   
         2      37500         2  2016-01-01 00:03:40   2016-01-01 00:12:47   
         3      37900         2  2016-01-01 00:05:38   2016-01-01 00:10:02   
         4      38500         1  2016-01-01 00:07:50   2016-01-01 00:23:42   
         
            passenger\_count  trip\_distance  pickup\_longitude  pickup\_latitude  \textbackslash{}
         0                2           1.20        -73.990578        40.732883   
         1                1           5.00        -73.994286        40.749153   
         2                6           2.54        -73.949821        40.785412   
         3                3           0.76        -74.002998        40.739220   
         4                1           2.40        -73.992546        40.766624   
         
            RatecodeID store\_and\_fwd\_flag  dropoff\_longitude  dropoff\_latitude  \textbackslash{}
         0           1                  N         -73.982307         40.747406   
         1           1                  N         -73.956688         40.747395   
         2           1                  N         -73.974586         40.758282   
         3           1                  N         -74.006714         40.744259   
         4           1                  N         -73.958771         40.763844   
         
            payment\_type  fare\_amount  extra  mta\_tax  tip\_amount  tolls\_amount  \textbackslash{}
         0             2          8.0    0.5      0.5        0.00           0.0   
         1             1         20.5    0.5      0.5        1.09           0.0   
         2             1          9.5    0.5      0.5        2.16           0.0   
         3             2          5.0    0.5      0.5        0.00           0.0   
         4             1         12.0    0.5      0.5        2.00           0.0   
         
            improvement\_surcharge  total\_amount  
         0                    0.3          9.30  
         1                    0.3         22.89  
         2                    0.3         12.96  
         3                    0.3          6.30  
         4                    0.3         15.30  
\end{Verbatim}
            
    \subsubsection{Question 3a}\label{question-3a}

Write a SQL Query to find all rides in \texttt{Jan16} that are less than
12 hours, or 0.5 days. We will use this query as a nested query
\texttt{q3a\_query} in question 3b.

\emph{Hint:} Ideas in \texttt{q1d\_query} can be heavily reused

    q1d\_query = f"""

SELECT *

\begin{verbatim}
        FROM {TABLE_NAME}
        
        WHERE tpep_pickup_datetime
            
            BETWEEN '2016-01-01' AND '2016-02-01'
            
            AND record_id % 100 == 0
        
        ORDER BY tpep_pickup_datetime
        
        """
\end{verbatim}

    \begin{Verbatim}[commandchars=\\\{\}]
{\color{incolor}In [{\color{incolor}33}]:} \PY{n}{q3a\PYZus{}query} \PY{o}{=} \PY{n}{f}\PY{l+s+s2}{\PYZdq{}\PYZdq{}\PYZdq{}}
         \PY{l+s+s2}{SELECT *}
         \PY{l+s+s2}{FROM (}\PY{l+s+si}{\PYZob{}q1d\PYZus{}query\PYZcb{}}\PY{l+s+s2}{)}
         \PY{l+s+s2}{WHERE (julianday(tpep\PYZus{}dropoff\PYZus{}datetime) \PYZhy{} julianday(tpep\PYZus{}pickup\PYZus{}datetime)) \PYZlt{} 0.5}
         \PY{l+s+s2}{            }\PY{l+s+s2}{\PYZdq{}\PYZdq{}\PYZdq{}}
         
         \PY{c+c1}{\PYZsh{} YOUR CODE HERE}
         \PY{c+c1}{\PYZsh{}\PYZsh{}raise NotImplementedError()}
         \PY{k}{with} \PY{n}{timeit}\PY{p}{(}\PY{p}{)}\PY{p}{:} \PY{c+c1}{\PYZsh{} should take \PYZlt{} 3 seconds}
             \PY{n}{q3a\PYZus{}df} \PY{o}{=} \PY{n}{pd}\PY{o}{.}\PY{n}{read\PYZus{}sql\PYZus{}query}\PY{p}{(}\PY{n}{q3a\PYZus{}query}\PY{p}{,} \PY{n}{sql\PYZus{}engine}\PY{p}{)}
         \PY{n}{q3a\PYZus{}df}\PY{o}{.}\PY{n}{head}\PY{p}{(}\PY{p}{)}
\end{Verbatim}


    \begin{Verbatim}[commandchars=\\\{\}]
2.47 s elapsed

    \end{Verbatim}

\begin{Verbatim}[commandchars=\\\{\}]
{\color{outcolor}Out[{\color{outcolor}33}]:}    record\_id  VendorID tpep\_pickup\_datetime tpep\_dropoff\_datetime  \textbackslash{}
         0      37300         1  2016-01-01 00:02:20   2016-01-01 00:11:58   
         1      37400         1  2016-01-01 00:03:04   2016-01-01 00:28:54   
         2      37500         2  2016-01-01 00:03:40   2016-01-01 00:12:47   
         3      37900         2  2016-01-01 00:05:38   2016-01-01 00:10:02   
         4      38500         1  2016-01-01 00:07:50   2016-01-01 00:23:42   
         
            passenger\_count  trip\_distance  pickup\_longitude  pickup\_latitude  \textbackslash{}
         0                2           1.20        -73.990578        40.732883   
         1                1           5.00        -73.994286        40.749153   
         2                6           2.54        -73.949821        40.785412   
         3                3           0.76        -74.002998        40.739220   
         4                1           2.40        -73.992546        40.766624   
         
            RatecodeID store\_and\_fwd\_flag  dropoff\_longitude  dropoff\_latitude  \textbackslash{}
         0           1                  N         -73.982307         40.747406   
         1           1                  N         -73.956688         40.747395   
         2           1                  N         -73.974586         40.758282   
         3           1                  N         -74.006714         40.744259   
         4           1                  N         -73.958771         40.763844   
         
            payment\_type  fare\_amount  extra  mta\_tax  tip\_amount  tolls\_amount  \textbackslash{}
         0             2          8.0    0.5      0.5        0.00           0.0   
         1             1         20.5    0.5      0.5        1.09           0.0   
         2             1          9.5    0.5      0.5        2.16           0.0   
         3             2          5.0    0.5      0.5        0.00           0.0   
         4             1         12.0    0.5      0.5        2.00           0.0   
         
            improvement\_surcharge  total\_amount  
         0                    0.3          9.30  
         1                    0.3         22.89  
         2                    0.3         12.96  
         3                    0.3          6.30  
         4                    0.3         15.30  
\end{Verbatim}
            
    \begin{Verbatim}[commandchars=\\\{\}]
{\color{incolor}In [{\color{incolor}34}]:} \PY{n}{cleaned\PYZus{}jan\PYZus{}16\PYZus{}df}\PY{p}{[}\PY{l+s+s1}{\PYZsq{}}\PY{l+s+s1}{duration}\PY{l+s+s1}{\PYZsq{}}\PY{p}{]} \PY{o}{=} \PY{n}{cleaned\PYZus{}jan\PYZus{}16\PYZus{}df}\PY{p}{[}\PY{l+s+s2}{\PYZdq{}}\PY{l+s+s2}{tpep\PYZus{}dropoff\PYZus{}datetime}\PY{l+s+s2}{\PYZdq{}}\PY{p}{]}\PY{o}{\PYZhy{}}\PY{n}{cleaned\PYZus{}jan\PYZus{}16\PYZus{}df}\PY{p}{[}\PY{l+s+s2}{\PYZdq{}}\PY{l+s+s2}{tpep\PYZus{}pickup\PYZus{}datetime}\PY{l+s+s2}{\PYZdq{}}\PY{p}{]}
         \PY{n}{cleaned\PYZus{}jan\PYZus{}16\PYZus{}df}\PY{p}{[}\PY{l+s+s1}{\PYZsq{}}\PY{l+s+s1}{duration}\PY{l+s+s1}{\PYZsq{}}\PY{p}{]} \PY{o}{=} \PY{n}{cleaned\PYZus{}jan\PYZus{}16\PYZus{}df}\PY{p}{[}\PY{l+s+s1}{\PYZsq{}}\PY{l+s+s1}{duration}\PY{l+s+s1}{\PYZsq{}}\PY{p}{]}\PY{o}{.}\PY{n}{dt}\PY{o}{.}\PY{n}{total\PYZus{}seconds}\PY{p}{(}\PY{p}{)}
         \PY{n}{cleaned\PYZus{}jan\PYZus{}16\PYZus{}df} \PY{o}{=} \PY{n}{cleaned\PYZus{}jan\PYZus{}16\PYZus{}df}\PY{p}{[}\PY{n}{cleaned\PYZus{}jan\PYZus{}16\PYZus{}df}\PY{p}{[}\PY{l+s+s1}{\PYZsq{}}\PY{l+s+s1}{duration}\PY{l+s+s1}{\PYZsq{}}\PY{p}{]} \PY{o}{\PYZlt{}} \PY{l+m+mi}{12} \PY{o}{*} \PY{l+m+mi}{3600}\PY{p}{]}
         \PY{k}{assert} \PY{n+nb}{len}\PY{p}{(}\PY{n}{q3a\PYZus{}df}\PY{p}{)} \PY{o}{==} \PY{n+nb}{len}\PY{p}{(}\PY{n}{cleaned\PYZus{}jan\PYZus{}16\PYZus{}df}\PY{p}{)}
\end{Verbatim}


    \begin{Verbatim}[commandchars=\\\{\}]
{\color{incolor}In [{\color{incolor}35}]:} \PY{n+nb}{len}\PY{p}{(}\PY{n}{q3a\PYZus{}df}\PY{p}{)}\PY{p}{,} \PY{n+nb}{len}\PY{p}{(}\PY{n}{cleaned\PYZus{}jan\PYZus{}16\PYZus{}df}\PY{p}{)}
\end{Verbatim}


\begin{Verbatim}[commandchars=\\\{\}]
{\color{outcolor}Out[{\color{outcolor}35}]:} (23642, 23642)
\end{Verbatim}
            
    \begin{Verbatim}[commandchars=\\\{\}]
{\color{incolor}In [{\color{incolor}36}]:} \PY{n}{cleaned\PYZus{}jan\PYZus{}16\PYZus{}df}\PY{o}{.}\PY{n}{head}\PY{p}{(}\PY{p}{)}
\end{Verbatim}


\begin{Verbatim}[commandchars=\\\{\}]
{\color{outcolor}Out[{\color{outcolor}36}]:}    record\_id  VendorID tpep\_pickup\_datetime tpep\_dropoff\_datetime  \textbackslash{}
         0      37300         1  2016-01-01 00:02:20   2016-01-01 00:11:58   
         1      37400         1  2016-01-01 00:03:04   2016-01-01 00:28:54   
         2      37500         2  2016-01-01 00:03:40   2016-01-01 00:12:47   
         3      37900         2  2016-01-01 00:05:38   2016-01-01 00:10:02   
         4      38500         1  2016-01-01 00:07:50   2016-01-01 00:23:42   
         
            passenger\_count  trip\_distance  pickup\_longitude  pickup\_latitude  \textbackslash{}
         0                2           1.20        -73.990578        40.732883   
         1                1           5.00        -73.994286        40.749153   
         2                6           2.54        -73.949821        40.785412   
         3                3           0.76        -74.002998        40.739220   
         4                1           2.40        -73.992546        40.766624   
         
            RatecodeID store\_and\_fwd\_flag    {\ldots}     dropoff\_latitude  payment\_type  \textbackslash{}
         0           1                  N    {\ldots}            40.747406             2   
         1           1                  N    {\ldots}            40.747395             1   
         2           1                  N    {\ldots}            40.758282             1   
         3           1                  N    {\ldots}            40.744259             2   
         4           1                  N    {\ldots}            40.763844             1   
         
            fare\_amount  extra  mta\_tax  tip\_amount  tolls\_amount  \textbackslash{}
         0          8.0    0.5      0.5        0.00           0.0   
         1         20.5    0.5      0.5        1.09           0.0   
         2          9.5    0.5      0.5        2.16           0.0   
         3          5.0    0.5      0.5        0.00           0.0   
         4         12.0    0.5      0.5        2.00           0.0   
         
            improvement\_surcharge  total\_amount  duration  
         0                    0.3          9.30     578.0  
         1                    0.3         22.89    1550.0  
         2                    0.3         12.96     547.0  
         3                    0.3          6.30     264.0  
         4                    0.3         15.30     952.0  
         
         [5 rows x 21 columns]
\end{Verbatim}
            
    \subsubsection{Question 3b}\label{question-3b}

Our objective is to predict the duration of taxi rides in the New York
City region. Therefore, we should verify that our dataset contains only
rides that are either starting or ending in New York (or are contained
within the NY region).

Based on different coordinate estimates of New York City, the
(inclusive) latitude and longitude ranges are (roughly) as follows:

\begin{itemize}
\tightlist
\item
  Latitude is between 40.63 and 40.85
\item
  Longitude is between -74.03 and -73.75
\end{itemize}

Write a SQL query to find all rides in \texttt{q3a\_query} that are
within the New York City region. We will use this query as a temporary
table \texttt{q3b\_query} in question 3c.

\begin{itemize}
\tightlist
\item
  Note: This query can be tedious to write. In practice people use
  special data types to encode geographical information. For example, if
  we were using Postgres (made in Berkeley!) instead of SQLite, we could
  use the geo-spatial data types provided as part of
  \href{https://postgis.net/}{PostGIS}.
\end{itemize}

\emph{Hint:} Ideas in \texttt{q3a\_query} can be heavily reused

    q3a\_query = f"""

SELECT *

FROM (\{q1d\_query\})

WHERE (julianday(tpep\_dropoff\_datetime) -
julianday(tpep\_pickup\_datetime)) \textless{} 0.5

Hint from f strings:

def to\_lowercase(input):

\begin{verbatim}
return input.lower()
\end{verbatim}

name = "Eric Idle"

f"\{to\_lowercase(name)\} is funny."

'eric idle is funny.'

    \begin{Verbatim}[commandchars=\\\{\}]
{\color{incolor}In [{\color{incolor}37}]:} \PY{c+c1}{\PYZsh{} Try using this function!}
         \PY{k}{def} \PY{n+nf}{bounding\PYZus{}condition}\PY{p}{(}\PY{n}{lat\PYZus{}l}\PY{p}{,} \PY{n}{lat\PYZus{}u}\PY{p}{,} \PY{n}{lon\PYZus{}l}\PY{p}{,} \PY{n}{lon\PYZus{}u}\PY{p}{)}\PY{p}{:}
             \PY{k}{return} \PY{n}{f}\PY{l+s+s2}{\PYZdq{}\PYZdq{}\PYZdq{}}
         \PY{l+s+s2}{            pickup\PYZus{}longitude \PYZlt{}= }\PY{l+s+si}{\PYZob{}lon\PYZus{}u\PYZcb{}}\PY{l+s+s2}{ AND}
         \PY{l+s+s2}{            pickup\PYZus{}longitude \PYZgt{}= }\PY{l+s+si}{\PYZob{}lon\PYZus{}l\PYZcb{}}\PY{l+s+s2}{ AND}
         \PY{l+s+s2}{            dropoff\PYZus{}longitude \PYZlt{}= }\PY{l+s+si}{\PYZob{}lon\PYZus{}u\PYZcb{}}\PY{l+s+s2}{ AND}
         \PY{l+s+s2}{            dropoff\PYZus{}longitude \PYZgt{}= }\PY{l+s+si}{\PYZob{}lon\PYZus{}l\PYZcb{}}\PY{l+s+s2}{ AND}
         \PY{l+s+s2}{            pickup\PYZus{}latitude \PYZlt{}= }\PY{l+s+si}{\PYZob{}lat\PYZus{}u\PYZcb{}}\PY{l+s+s2}{ AND}
         \PY{l+s+s2}{            pickup\PYZus{}latitude \PYZgt{}= }\PY{l+s+si}{\PYZob{}lat\PYZus{}l\PYZcb{}}\PY{l+s+s2}{ AND}
         \PY{l+s+s2}{            dropoff\PYZus{}latitude \PYZlt{}= }\PY{l+s+si}{\PYZob{}lat\PYZus{}u\PYZcb{}}\PY{l+s+s2}{ AND}
         \PY{l+s+s2}{            dropoff\PYZus{}latitude \PYZgt{}= }\PY{l+s+si}{\PYZob{}lat\PYZus{}l\PYZcb{}}\PY{l+s+s2}{ }
         \PY{l+s+s2}{            }\PY{l+s+s2}{\PYZdq{}\PYZdq{}\PYZdq{}}
         
         \PY{n}{q3b\PYZus{}query} \PY{o}{=} \PY{n}{f}\PY{l+s+s2}{\PYZdq{}\PYZdq{}\PYZdq{}}
         \PY{l+s+s2}{SELECT *}
         \PY{l+s+s2}{FROM (}\PY{l+s+si}{\PYZob{}q3a\PYZus{}query\PYZcb{}}\PY{l+s+s2}{)}
         \PY{l+s+s2}{WHERE }\PY{l+s+s2}{\PYZob{}}\PY{l+s+s2}{bounding\PYZus{}condition(40.63, 40.85, \PYZhy{}74.03, \PYZhy{}73.75)\PYZcb{}}
         \PY{l+s+s2}{            }\PY{l+s+s2}{\PYZdq{}\PYZdq{}\PYZdq{}}
         \PY{n}{lat\PYZus{}l} \PY{o}{=} \PY{l+m+mf}{40.63}
         \PY{n}{lat\PYZus{}u} \PY{o}{=} \PY{l+m+mf}{40.85}
         \PY{n}{lon\PYZus{}l} \PY{o}{=} \PY{o}{\PYZhy{}}\PY{l+m+mf}{74.03}
         \PY{n}{lon\PYZus{}u} \PY{o}{=} \PY{o}{\PYZhy{}}\PY{l+m+mf}{73.75}
         
         \PY{c+c1}{\PYZsh{} YOUR CODE HERE}
         \PY{c+c1}{\PYZsh{}\PYZsh{}raise NotImplementedError()}
         \PY{k}{with} \PY{n}{timeit}\PY{p}{(}\PY{p}{)}\PY{p}{:} \PY{c+c1}{\PYZsh{} should take \PYZlt{} 3 seconds}
             \PY{n}{q3b\PYZus{}df} \PY{o}{=} \PY{n}{pd}\PY{o}{.}\PY{n}{read\PYZus{}sql\PYZus{}query}\PY{p}{(}\PY{n}{q3b\PYZus{}query}\PY{p}{,} \PY{n}{sql\PYZus{}engine}\PY{p}{)}
         \PY{n}{q3b\PYZus{}df}\PY{o}{.}\PY{n}{head}\PY{p}{(}\PY{p}{)}
\end{Verbatim}


    \begin{Verbatim}[commandchars=\\\{\}]
2.49 s elapsed

    \end{Verbatim}

\begin{Verbatim}[commandchars=\\\{\}]
{\color{outcolor}Out[{\color{outcolor}37}]:}    record\_id  VendorID tpep\_pickup\_datetime tpep\_dropoff\_datetime  \textbackslash{}
         0      37300         1  2016-01-01 00:02:20   2016-01-01 00:11:58   
         1      37400         1  2016-01-01 00:03:04   2016-01-01 00:28:54   
         2      37500         2  2016-01-01 00:03:40   2016-01-01 00:12:47   
         3      37900         2  2016-01-01 00:05:38   2016-01-01 00:10:02   
         4      38500         1  2016-01-01 00:07:50   2016-01-01 00:23:42   
         
            passenger\_count  trip\_distance  pickup\_longitude  pickup\_latitude  \textbackslash{}
         0                2           1.20        -73.990578        40.732883   
         1                1           5.00        -73.994286        40.749153   
         2                6           2.54        -73.949821        40.785412   
         3                3           0.76        -74.002998        40.739220   
         4                1           2.40        -73.992546        40.766624   
         
            RatecodeID store\_and\_fwd\_flag  dropoff\_longitude  dropoff\_latitude  \textbackslash{}
         0           1                  N         -73.982307         40.747406   
         1           1                  N         -73.956688         40.747395   
         2           1                  N         -73.974586         40.758282   
         3           1                  N         -74.006714         40.744259   
         4           1                  N         -73.958771         40.763844   
         
            payment\_type  fare\_amount  extra  mta\_tax  tip\_amount  tolls\_amount  \textbackslash{}
         0             2          8.0    0.5      0.5        0.00           0.0   
         1             1         20.5    0.5      0.5        1.09           0.0   
         2             1          9.5    0.5      0.5        2.16           0.0   
         3             2          5.0    0.5      0.5        0.00           0.0   
         4             1         12.0    0.5      0.5        2.00           0.0   
         
            improvement\_surcharge  total\_amount  
         0                    0.3          9.30  
         1                    0.3         22.89  
         2                    0.3         12.96  
         3                    0.3          6.30  
         4                    0.3         15.30  
\end{Verbatim}
            
    By contrast, the approach two (pandas) equivalent is given below.

    \begin{Verbatim}[commandchars=\\\{\}]
{\color{incolor}In [{\color{incolor}38}]:} \PY{n}{cleaned\PYZus{}jan\PYZus{}16\PYZus{}df} \PY{o}{=} \PY{n}{cleaned\PYZus{}jan\PYZus{}16\PYZus{}df}\PY{p}{[}\PY{n}{cleaned\PYZus{}jan\PYZus{}16\PYZus{}df}\PY{p}{[}\PY{l+s+s1}{\PYZsq{}}\PY{l+s+s1}{pickup\PYZus{}longitude}\PY{l+s+s1}{\PYZsq{}}\PY{p}{]} \PY{o}{\PYZlt{}}\PY{o}{=} \PY{o}{\PYZhy{}}\PY{l+m+mf}{73.75}\PY{p}{]}
         \PY{n}{cleaned\PYZus{}jan\PYZus{}16\PYZus{}df} \PY{o}{=} \PY{n}{cleaned\PYZus{}jan\PYZus{}16\PYZus{}df}\PY{p}{[}\PY{n}{cleaned\PYZus{}jan\PYZus{}16\PYZus{}df}\PY{p}{[}\PY{l+s+s1}{\PYZsq{}}\PY{l+s+s1}{pickup\PYZus{}longitude}\PY{l+s+s1}{\PYZsq{}}\PY{p}{]} \PY{o}{\PYZgt{}}\PY{o}{=} \PY{o}{\PYZhy{}}\PY{l+m+mf}{74.03}\PY{p}{]}
         \PY{n}{cleaned\PYZus{}jan\PYZus{}16\PYZus{}df} \PY{o}{=} \PY{n}{cleaned\PYZus{}jan\PYZus{}16\PYZus{}df}\PY{p}{[}\PY{n}{cleaned\PYZus{}jan\PYZus{}16\PYZus{}df}\PY{p}{[}\PY{l+s+s1}{\PYZsq{}}\PY{l+s+s1}{pickup\PYZus{}latitude}\PY{l+s+s1}{\PYZsq{}}\PY{p}{]} \PY{o}{\PYZlt{}}\PY{o}{=} \PY{l+m+mf}{40.85}\PY{p}{]}
         \PY{n}{cleaned\PYZus{}jan\PYZus{}16\PYZus{}df} \PY{o}{=} \PY{n}{cleaned\PYZus{}jan\PYZus{}16\PYZus{}df}\PY{p}{[}\PY{n}{cleaned\PYZus{}jan\PYZus{}16\PYZus{}df}\PY{p}{[}\PY{l+s+s1}{\PYZsq{}}\PY{l+s+s1}{pickup\PYZus{}latitude}\PY{l+s+s1}{\PYZsq{}}\PY{p}{]} \PY{o}{\PYZgt{}}\PY{o}{=} \PY{l+m+mf}{40.63}\PY{p}{]}
         \PY{n}{cleaned\PYZus{}jan\PYZus{}16\PYZus{}df} \PY{o}{=} \PY{n}{cleaned\PYZus{}jan\PYZus{}16\PYZus{}df}\PY{p}{[}\PY{n}{cleaned\PYZus{}jan\PYZus{}16\PYZus{}df}\PY{p}{[}\PY{l+s+s1}{\PYZsq{}}\PY{l+s+s1}{dropoff\PYZus{}longitude}\PY{l+s+s1}{\PYZsq{}}\PY{p}{]} \PY{o}{\PYZlt{}}\PY{o}{=} \PY{o}{\PYZhy{}}\PY{l+m+mf}{73.75}\PY{p}{]}
         \PY{n}{cleaned\PYZus{}jan\PYZus{}16\PYZus{}df} \PY{o}{=} \PY{n}{cleaned\PYZus{}jan\PYZus{}16\PYZus{}df}\PY{p}{[}\PY{n}{cleaned\PYZus{}jan\PYZus{}16\PYZus{}df}\PY{p}{[}\PY{l+s+s1}{\PYZsq{}}\PY{l+s+s1}{dropoff\PYZus{}longitude}\PY{l+s+s1}{\PYZsq{}}\PY{p}{]} \PY{o}{\PYZgt{}}\PY{o}{=} \PY{o}{\PYZhy{}}\PY{l+m+mf}{74.03}\PY{p}{]}
         \PY{n}{cleaned\PYZus{}jan\PYZus{}16\PYZus{}df} \PY{o}{=} \PY{n}{cleaned\PYZus{}jan\PYZus{}16\PYZus{}df}\PY{p}{[}\PY{n}{cleaned\PYZus{}jan\PYZus{}16\PYZus{}df}\PY{p}{[}\PY{l+s+s1}{\PYZsq{}}\PY{l+s+s1}{dropoff\PYZus{}latitude}\PY{l+s+s1}{\PYZsq{}}\PY{p}{]} \PY{o}{\PYZlt{}}\PY{o}{=} \PY{l+m+mf}{40.85}\PY{p}{]}
         \PY{n}{cleaned\PYZus{}jan\PYZus{}16\PYZus{}df} \PY{o}{=} \PY{n}{cleaned\PYZus{}jan\PYZus{}16\PYZus{}df}\PY{p}{[}\PY{n}{cleaned\PYZus{}jan\PYZus{}16\PYZus{}df}\PY{p}{[}\PY{l+s+s1}{\PYZsq{}}\PY{l+s+s1}{dropoff\PYZus{}latitude}\PY{l+s+s1}{\PYZsq{}}\PY{p}{]} \PY{o}{\PYZgt{}}\PY{o}{=} \PY{l+m+mf}{40.63}\PY{p}{]}
         \PY{k}{assert} \PY{n+nb}{len}\PY{p}{(}\PY{n}{q3b\PYZus{}df}\PY{p}{)} \PY{o}{==} \PY{n+nb}{len}\PY{p}{(}\PY{n}{cleaned\PYZus{}jan\PYZus{}16\PYZus{}df}\PY{p}{)}
\end{Verbatim}


    \begin{Verbatim}[commandchars=\\\{\}]
{\color{incolor}In [{\color{incolor}39}]:} \PY{n+nb}{len}\PY{p}{(}\PY{n}{q3b\PYZus{}df}\PY{p}{)}\PY{p}{,} \PY{n+nb}{len}\PY{p}{(}\PY{n}{cleaned\PYZus{}jan\PYZus{}16\PYZus{}df}\PY{p}{)}
\end{Verbatim}


\begin{Verbatim}[commandchars=\\\{\}]
{\color{outcolor}Out[{\color{outcolor}39}]:} (22944, 22944)
\end{Verbatim}
            
    \subsubsection{Question 3c}\label{question-3c}

The \texttt{passenger\_count} variable has a minimum value of 0
passengers and a maximum value of 9 passengers. Having 0 passengers does
not make sense in the context of this business case; it is likely an
error and should therefore be removed from our dataset.

Write a SQL query to find all rides in \texttt{q3b\_query} with
\texttt{passenger\_count} greater than 0.

\emph{Hint:} Ideas in \texttt{q3b\_query} can be heavily reused

    Use q3b\_query!

def bounding\_condition(lat\_l, lat\_u, lon\_l, lon\_u): return f"""
pickup\_longitude \textless{}= \{lon\_u\} AND pickup\_longitude
\textgreater{}= \{lon\_l\} AND dropoff\_longitude \textless{}=
\{lon\_u\} AND dropoff\_longitude \textgreater{}= \{lon\_l\} AND
pickup\_latitude \textless{}= \{lat\_u\} AND pickup\_latitude
\textgreater{}= \{lat\_l\} AND dropoff\_latitude \textless{}= \{lat\_u\}
AND dropoff\_latitude \textgreater{}= \{lat\_l\} """

q3b\_query = f"""

SELECT *

FROM (\{q3a\_query\})

WHERE \{bounding\_condition(40.63, 40.85, -74.03, -73.75)\}

    \begin{Verbatim}[commandchars=\\\{\}]
{\color{incolor}In [{\color{incolor}40}]:} \PY{k}{def} \PY{n+nf}{passenger\PYZus{}condition}\PY{p}{(}\PY{n}{passenger\PYZus{}amount}\PY{p}{)}\PY{p}{:}
             \PY{k}{return} \PY{n}{f}\PY{l+s+s2}{\PYZdq{}\PYZdq{}\PYZdq{}}
         \PY{l+s+s2}{        passenger\PYZus{}count \PYZgt{} }\PY{l+s+si}{\PYZob{}passenger\PYZus{}amount\PYZcb{}}
         \PY{l+s+s2}{        }\PY{l+s+s2}{\PYZdq{}\PYZdq{}\PYZdq{}}
         
         
         \PY{n}{q3c\PYZus{}query} \PY{o}{=} \PY{n}{f}\PY{l+s+s2}{\PYZdq{}\PYZdq{}\PYZdq{}}
         \PY{l+s+s2}{SELECT *}
         \PY{l+s+s2}{FROM (}\PY{l+s+si}{\PYZob{}q3b\PYZus{}query\PYZcb{}}\PY{l+s+s2}{)}
         \PY{l+s+s2}{WHERE }\PY{l+s+s2}{\PYZob{}}\PY{l+s+s2}{passenger\PYZus{}condition(0)\PYZcb{}}
         \PY{l+s+s2}{            }\PY{l+s+s2}{\PYZdq{}\PYZdq{}\PYZdq{}}
         
         \PY{c+c1}{\PYZsh{} YOUR CODE HERE}
         \PY{c+c1}{\PYZsh{}\PYZsh{}raise NotImplementedError()}
         \PY{k}{with} \PY{n}{timeit}\PY{p}{(}\PY{p}{)}\PY{p}{:}
             \PY{n}{q3c\PYZus{}df} \PY{o}{=} \PY{n}{pd}\PY{o}{.}\PY{n}{read\PYZus{}sql\PYZus{}query}\PY{p}{(}\PY{n}{q3c\PYZus{}query}\PY{p}{,} \PY{n}{sql\PYZus{}engine}\PY{p}{)}
         \PY{n}{q3c\PYZus{}df}\PY{o}{.}\PY{n}{head}\PY{p}{(}\PY{p}{)}
\end{Verbatim}


    \begin{Verbatim}[commandchars=\\\{\}]
2.46 s elapsed

    \end{Verbatim}

\begin{Verbatim}[commandchars=\\\{\}]
{\color{outcolor}Out[{\color{outcolor}40}]:}    record\_id  VendorID tpep\_pickup\_datetime tpep\_dropoff\_datetime  \textbackslash{}
         0      37300         1  2016-01-01 00:02:20   2016-01-01 00:11:58   
         1      37400         1  2016-01-01 00:03:04   2016-01-01 00:28:54   
         2      37500         2  2016-01-01 00:03:40   2016-01-01 00:12:47   
         3      37900         2  2016-01-01 00:05:38   2016-01-01 00:10:02   
         4      38500         1  2016-01-01 00:07:50   2016-01-01 00:23:42   
         
            passenger\_count  trip\_distance  pickup\_longitude  pickup\_latitude  \textbackslash{}
         0                2           1.20        -73.990578        40.732883   
         1                1           5.00        -73.994286        40.749153   
         2                6           2.54        -73.949821        40.785412   
         3                3           0.76        -74.002998        40.739220   
         4                1           2.40        -73.992546        40.766624   
         
            RatecodeID store\_and\_fwd\_flag  dropoff\_longitude  dropoff\_latitude  \textbackslash{}
         0           1                  N         -73.982307         40.747406   
         1           1                  N         -73.956688         40.747395   
         2           1                  N         -73.974586         40.758282   
         3           1                  N         -74.006714         40.744259   
         4           1                  N         -73.958771         40.763844   
         
            payment\_type  fare\_amount  extra  mta\_tax  tip\_amount  tolls\_amount  \textbackslash{}
         0             2          8.0    0.5      0.5        0.00           0.0   
         1             1         20.5    0.5      0.5        1.09           0.0   
         2             1          9.5    0.5      0.5        2.16           0.0   
         3             2          5.0    0.5      0.5        0.00           0.0   
         4             1         12.0    0.5      0.5        2.00           0.0   
         
            improvement\_surcharge  total\_amount  
         0                    0.3          9.30  
         1                    0.3         22.89  
         2                    0.3         12.96  
         3                    0.3          6.30  
         4                    0.3         15.30  
\end{Verbatim}
            
    \begin{Verbatim}[commandchars=\\\{\}]
{\color{incolor}In [{\color{incolor}41}]:} \PY{n}{cleaned\PYZus{}jan\PYZus{}16\PYZus{}df} \PY{o}{=} \PY{n}{cleaned\PYZus{}jan\PYZus{}16\PYZus{}df}\PY{p}{[}\PY{n}{cleaned\PYZus{}jan\PYZus{}16\PYZus{}df}\PY{p}{[}\PY{l+s+s1}{\PYZsq{}}\PY{l+s+s1}{passenger\PYZus{}count}\PY{l+s+s1}{\PYZsq{}}\PY{p}{]} \PY{o}{\PYZgt{}} \PY{l+m+mi}{0}\PY{p}{]}
         \PY{k}{assert} \PY{n+nb}{len}\PY{p}{(}\PY{n}{q3c\PYZus{}df}\PY{p}{)} \PY{o}{==} \PY{n+nb}{len}\PY{p}{(}\PY{n}{cleaned\PYZus{}jan\PYZus{}16\PYZus{}df}\PY{p}{)}
\end{Verbatim}


    \subsubsection{Question 3d}\label{question-3d}

If you passed all the previous tests, then we are done cleaning! We
would like to check how many records we have removed to ensure that it
is a relatively small number (otherwise we might introduce bias within
our dataset). In the cell below calculate the number and proportion of
records we removed from the original \texttt{jan\_16\_df} during the
data cleaning process.

To avoid possible error propagation, you should use our
\texttt{cleaned\_jan\_16\_df} in your solution as the final cleaned
dataset instead of your \texttt{q3c\_df}.

    \begin{Verbatim}[commandchars=\\\{\}]
{\color{incolor}In [{\color{incolor}42}]:} \PY{n}{num\PYZus{}records\PYZus{}removed} \PY{o}{=} \PY{n}{jan\PYZus{}16\PYZus{}df}\PY{o}{.}\PY{n}{shape}\PY{p}{[}\PY{l+m+mi}{0}\PY{p}{]} \PY{o}{\PYZhy{}} \PY{n}{cleaned\PYZus{}jan\PYZus{}16\PYZus{}df}\PY{o}{.}\PY{n}{shape}\PY{p}{[}\PY{l+m+mi}{0}\PY{p}{]}
         \PY{n}{proportion\PYZus{}records\PYZus{}removed} \PY{o}{=} \PY{n}{num\PYZus{}records\PYZus{}removed} \PY{o}{/} \PY{n}{jan\PYZus{}16\PYZus{}df}\PY{o}{.}\PY{n}{shape}\PY{p}{[}\PY{l+m+mi}{0}\PY{p}{]}
         
         \PY{c+c1}{\PYZsh{} YOUR CODE HERE}
         \PY{c+c1}{\PYZsh{}\PYZsh{}raise NotImplementedError()}
         
         \PY{n+nb}{print}\PY{p}{(}\PY{n}{f}\PY{l+s+s1}{\PYZsq{}}\PY{l+s+s1}{Records removed:}\PY{l+s+si}{\PYZob{}num\PYZus{}records\PYZus{}removed\PYZcb{}}\PY{l+s+s1}{\PYZsq{}}\PY{p}{)}
         \PY{n+nb}{print}\PY{p}{(}\PY{n}{f}\PY{l+s+s1}{\PYZsq{}}\PY{l+s+s1}{Proportion records removed:}\PY{l+s+si}{\PYZob{}proportion\PYZus{}records\PYZus{}removed\PYZcb{}}\PY{l+s+s1}{\PYZsq{}}\PY{p}{)}
\end{Verbatim}


    \begin{Verbatim}[commandchars=\\\{\}]
Records removed:731
Proportion records removed:0.030877756188223367

    \end{Verbatim}

    \begin{Verbatim}[commandchars=\\\{\}]
{\color{incolor}In [{\color{incolor}43}]:} \PY{k}{assert} \PY{n}{proportion\PYZus{}records\PYZus{}removed} \PY{o}{\PYZlt{}} \PY{l+m+mf}{0.04}
         \PY{k}{assert} \PY{n}{proportion\PYZus{}records\PYZus{}removed} \PY{o}{\PYZgt{}} \PY{l+m+mf}{0.03}
\end{Verbatim}


    At this point, let's take a look at the final query that cleaned up the
data. Nesting SQL queries or creating views for future re-use are common
pattern in analytical queries. Pay attention to each WHERE clause.

    \begin{Verbatim}[commandchars=\\\{\}]
{\color{incolor}In [{\color{incolor}44}]:} \PY{n+nb}{print}\PY{p}{(}\PY{n}{q3c\PYZus{}query}\PY{p}{)}
\end{Verbatim}


    \begin{Verbatim}[commandchars=\\\{\}]

SELECT *
FROM (
SELECT *
FROM (
SELECT *
FROM (
SELECT * 
            FROM taxi
            WHERE tpep\_pickup\_datetime
                BETWEEN '2016-01-01' AND '2016-02-01'
                AND record\_id \% 100 == 0
            ORDER BY tpep\_pickup\_datetime
            )
WHERE (julianday(tpep\_dropoff\_datetime) - julianday(tpep\_pickup\_datetime)) < 0.5
            )
WHERE 
            pickup\_longitude <= -73.75 AND
            pickup\_longitude >= -74.03 AND
            dropoff\_longitude <= -73.75 AND
            dropoff\_longitude >= -74.03 AND
            pickup\_latitude <= 40.85 AND
            pickup\_latitude >= 40.63 AND
            dropoff\_latitude <= 40.85 AND
            dropoff\_latitude >= 40.63 
            
            )
WHERE 
        passenger\_count > 0
        
            

    \end{Verbatim}

    \subsection{Question 4: Training and Validation
Split}\label{question-4-training-and-validation-split}

Now that we have fetched and cleaned our data, let's create training and
validation sets. We will use a 80/20 ratio for training/validation and
set \texttt{random\_state=42} for the purpose of grading.

    \begin{Verbatim}[commandchars=\\\{\}]
{\color{incolor}In [{\color{incolor}45}]:} \PY{k+kn}{from} \PY{n+nn}{sklearn}\PY{n+nn}{.}\PY{n+nn}{model\PYZus{}selection} \PY{k}{import} \PY{n}{train\PYZus{}test\PYZus{}split}
         \PY{n}{train\PYZus{}df}\PY{p}{,} \PY{n}{val\PYZus{}df} \PY{o}{=} \PY{n}{train\PYZus{}test\PYZus{}split}\PY{p}{(}\PY{n}{cleaned\PYZus{}jan\PYZus{}16\PYZus{}df}\PY{p}{,} \PY{n}{test\PYZus{}size}\PY{o}{=}\PY{l+m+mf}{0.2}\PY{p}{,} \PY{n}{random\PYZus{}state}\PY{o}{=}\PY{l+m+mi}{42}\PY{p}{)}
\end{Verbatim}


    \begin{Verbatim}[commandchars=\\\{\}]
{\color{incolor}In [{\color{incolor}46}]:} \PY{c+c1}{\PYZsh{} Check that 80\PYZpc{} records in training and 20\PYZpc{} in validation set.}
         \PY{k}{assert} \PY{n+nb}{len}\PY{p}{(}\PY{n}{train\PYZus{}df}\PY{p}{)} \PY{o}{\PYZlt{}} \PY{l+m+mi}{18500}
         \PY{k}{assert} \PY{n+nb}{len}\PY{p}{(}\PY{n}{train\PYZus{}df}\PY{p}{)} \PY{o}{\PYZgt{}} \PY{l+m+mi}{17000}
         \PY{k}{assert} \PY{n+nb}{len}\PY{p}{(}\PY{n}{val\PYZus{}df}\PY{p}{)} \PY{o}{\PYZgt{}} \PY{l+m+mi}{4000}
         \PY{k}{assert} \PY{n+nb}{len}\PY{p}{(}\PY{n}{val\PYZus{}df}\PY{p}{)} \PY{o}{\PYZlt{}} \PY{l+m+mi}{5000}
\end{Verbatim}


    \subsection{Part 1 Exports}\label{part-1-exports}

    Throughout our analysis, we have formatted and cleaned our data. Since
we are ready to begin the feature engineering process, a good practice
is to start a new notebook (since this one is getting quite long!). Now,
we will save our formatted data, which we will load in part 2.
\textbf{Be sure to run the cell below!}

Please read the documentation below on saving and loading hdf files.

https://pandas.pydata.org/pandas-docs/stable/generated/pandas.DataFrame.to\_hdf.html

https://pandas.pydata.org/pandas-docs/version/0.22/generated/pandas.read\_hdf.html

    \begin{Verbatim}[commandchars=\\\{\}]
{\color{incolor}In [{\color{incolor}47}]:} \PY{n}{Path}\PY{p}{(}\PY{l+s+s2}{\PYZdq{}}\PY{l+s+s2}{data/part1}\PY{l+s+s2}{\PYZdq{}}\PY{p}{)}\PY{o}{.}\PY{n}{mkdir}\PY{p}{(}\PY{n}{parents}\PY{o}{=}\PY{k+kc}{True}\PY{p}{,} \PY{n}{exist\PYZus{}ok}\PY{o}{=}\PY{k+kc}{True}\PY{p}{)}
         \PY{n}{data\PYZus{}file} \PY{o}{=} \PY{n}{Path}\PY{p}{(}\PY{l+s+s2}{\PYZdq{}}\PY{l+s+s2}{data/part1}\PY{l+s+s2}{\PYZdq{}}\PY{p}{,} \PY{l+s+s2}{\PYZdq{}}\PY{l+s+s2}{cleaned\PYZus{}data.hdf}\PY{l+s+s2}{\PYZdq{}}\PY{p}{)} \PY{c+c1}{\PYZsh{} Path of hdf file}
         \PY{n}{train\PYZus{}df}\PY{o}{.}\PY{n}{to\PYZus{}hdf}\PY{p}{(}\PY{n}{data\PYZus{}file}\PY{p}{,} \PY{l+s+s2}{\PYZdq{}}\PY{l+s+s2}{train}\PY{l+s+s2}{\PYZdq{}}\PY{p}{)} \PY{c+c1}{\PYZsh{} Train data of hdf file}
         \PY{n}{val\PYZus{}df}\PY{o}{.}\PY{n}{to\PYZus{}hdf}\PY{p}{(}\PY{n}{data\PYZus{}file}\PY{p}{,} \PY{l+s+s2}{\PYZdq{}}\PY{l+s+s2}{val}\PY{l+s+s2}{\PYZdq{}}\PY{p}{)} \PY{c+c1}{\PYZsh{} Val data of hdf file}
\end{Verbatim}


    \subsection{Part 1 Conclusions}\label{part-1-conclusions}

    We have downloaded/loaded our data, cleaned the data, and split our data
into a training and test set to use in future analysis and modeling.

    \textbf{Please proceed to part 2, where we will be exploring the taxi
ride training set.}

    \subsection{Submission}\label{submission}

You're almost done!

Before submitting this assignment, ensure that you have:

\begin{enumerate}
\def\labelenumi{\arabic{enumi}.}
\tightlist
\item
  Restarted the Kernel (in the menubar, select
  Kernel\(\rightarrow\)Restart \& Run All)
\item
  Validated the notebook by clicking the "Validate" button.
\end{enumerate}

Then,

\begin{enumerate}
\def\labelenumi{\arabic{enumi}.}
\tightlist
\item
  \textbf{Submit} the assignment via the Assignments tab in
  \textbf{Datahub}
\item
  \textbf{Upload and tag} the manually reviewed portions of the
  assignment on \textbf{Gradescope}
\end{enumerate}


    % Add a bibliography block to the postdoc
    
    
    
    \end{document}
